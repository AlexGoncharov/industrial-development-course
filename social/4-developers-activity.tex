\documentclass{../industrial-development}
\graphicspath{{4-developers-activity/}}

\title{Деятельность разработчика в компаниях ИТ-индустрии}
\author{Соколов Роман, Садикова Анастасия}
\date{}

\begin{document}

\begin{frame}
  \titlepage
\end{frame}


\section{Разработчик в узком и широком смыслах}
\subsection{Определение разработчика}
\begin{frame} \frametitle{Определение разработчика}
	\begin{minipage}{0.47\textwidth}
		\begin{block}{Разработчик в узком смысле}
			Человек, который занимается написанием кода (программированием) и,~иногда, его тестированием
		\end{block}
	\end{minipage}
	\hfill
	\begin{minipage}{0.5\textwidth}
		\begin{block}{Разработчик в широком смысле}
			Человек, который занимается разработкой ПО (т.е. его деятельность затрагивает почти все фазы жизненного цикла продукта)
		\end{block}
	\end{minipage}
	
\end{frame}
\lecturenotes
\\В узком смысле программирование рассматривается как кодирование --- реализация в виде программы одного или нескольких взаимосвязанных алгоритмов (в современных условиях это осуществляется с применением языков программирования).//
В более широком смысле процесс программирования охватывает создание и моделирование (то есть разработку) алгоритмов и анализ потребностей будущих пользователей программного обеспечения. В его число обязанностей может входить: практически все фазы жизненного цикла (но в той степени, в которой это касается разработчика), самостоятельная постановка задач, высокоуровневая картина продукта, широкая кооперация с коллегами, возможно даже взаимодействие с заказчиком.


\subsection{Примеры разработчика}
\begin{frame} \frametitle{Примеры разработчика}
	\begin{minipage}{0.47\textwidth}
		\textbf{В узком смысле:}
		
		Обычный рядовой программист, выполнляющий поставленные ему задачи.
	\end{minipage}
\hfill
	\begin{minipage}{0.5\textwidth}
		\textbf{В широком смысле:}
		
		Тимлид, задачей которого является организация процесса разработки ПО, приводящий к его завершению и релизу.
	\end{minipage}
\end{frame}
\lecturenotes
В узком смысле идеальным примером разработчика ПО является обычный рядовой программист или даже стажер. Ему поступает строго определенная задача от тимлда, начальника, наставника и т.п. и он точно выполняет, проверяет и сдает её.//
В широком смысле примером разработчика является тимлид. Это высококвалифицированный профессионал, который смотрит на разработку ПО сверху (обладает высокоуровневой картиной продукта), разрабатывает архитектуру будущего продукта, оценивает и распределяет задачи между остальными коллегами. Если рядовой программист выполняет задачи, то тимлид - ставит их.


\section{Типичные задачи разработчика ПО}
\subsection{Обучение}
\begin{frame} \frametitle{Определение обучения}
		\begin{minipage}{0.47\textwidth}
			\begin{block}{В узком смысле}
				Процесс изучения новых технологий с целью поддержания минимального уровня квалификации, необходимого для работы над~поставленной задачей
			\end{block}
		\end{minipage}
		\hfill
		\begin{minipage}{0.47\textwidth}
			\begin{block}{В широком смысле}
				Плановый процесс изучения новых технологий (для решения текущей задачи или проектов в~будущем) или общее повышение квалификации в~рамках компании
			\end{block}
		\end{minipage}
\end{frame}
\lecturenotes
\\Обучение в ИТ-компании это обязательный процесс для каждого разработчика, поскольку необходимый уровень квалификации для выполнения новых задач динамически меняется.\\
Как правило, будущие разработчики сперва нанимаются на должность стажера, которым выделяется наставник --- человек, непосредственно модерирующий процесс обучения и помогающий разобраться с возникшими (как правило типичными) проблемами.\\
Часто хорошие компании устраивают мероприятия по обмену общими знаниями, где от разработчика требуется как делиться своим опытом с другими сотрудниками, так и самому приобретать новые знания. Данный процесс так же включает в себя и самостоятельное изучение новых областей разработки, методик, языков и т.д., но это остается на совести самого разработчика.\\
В отличии от узкого смысла, в широком смысле обучение разработчика проводится не для поддержания минимального уровня квалификации, а для развития в целом и продвижения по карьерной лестнице.\\
Благодаря общему взгляду «сверху» на проект, разработчик может заранее понять какие новые технологии возможно применить или потребуется применить в текущем проекте и предварительно изучить их самостоятельно, либо назначить изучение другому разработчику, которому предстоит решение данной задачи.

\begin{frame} \frametitle{Примеры обучения}
		\begin{minipage}{0.48\textwidth}
			\textbf{В узком смысле:}
			
				Примером обучения в узком смысле является выполнение заданий стажера от наставника; посещение митапов, курсов, мероприятий
		\end{minipage}
		\hfill
		\begin{minipage}{0.48\textwidth}
			\textbf{В широком смысле:}
			
				Пример обучения в широком смысле --- система грейдов в рамках компании и самостоятельное изучение новых областей программирования
		\end{minipage}
\end{frame}
\lecturenotes
Чтобы новый сотрудник мог влиться в общий процесс разработки ПО, обычно его курирует наставник. Выполняя его задания, стажер получает минимальные знания для выполнения будущих задач.\\
Посещение митапов и курсов для разработчиков, в основном, проводятся для рассказа о новых технологиях, освоенных другими разработчиками, их опыта применения и с какими трудностями они столкнулись. Часто в компаниях проводятся сертификации и экзамены с целью повышения текущего уровня знаний разработчика.\\
Система грейдов --- это система уровней знаний разработчиков в компании. Для присвоения присвоения некого уровня нужно пройти собеседование и подтвердить на нём соответствие текущего уровня знаний для следующего грейда, который постоянно возрастает.\\
В целом, разработчикам, которые хотят оставаться актуальными в своей профессии, необходимо систематически самостоятельно изучать новые технологии своей среды.

\subsection{Разработка архитектуры ПО}
\begin{frame} \frametitle{Определение разработки архитектуры ПО}
  \begin{block}{Разработка архитектуры ПО}
  	Процесс предварительного выбора и утверждения будущей логики разрабатываемого ПО текущим руководителем проекта или коллективом разработчиков. Этот процесс возможен только в широком смысле, поскольку требует взгляда на проект сверху, не доступного рядовым разработчикам 
  \end{block}{}	
\end{frame}
\lecturenotes
\\Разработка архитектуры ПО --- процесс подбора наиболее подходящей парадигмы разработки программы для конкретной поставленной задачи. Перед непосредственным написанием кода ПО, ведущему разработчику необходимо создать концепцию общей работы структур данных и логики. Процесс выбора архитектуры заключается в подборе наиболее оптимальной парадигмы, как правило, на основании опыта участвующих в процессе разработчиков.
Основными факторами выбора архитектуры являются риски: недостаточный функционал, переносимость, способность команды справиться с задачей, недостаточная оптимизация, некорректная выполнение задач, низкая модифицируемость и т.д. Как правило, решение принимается на общем обсуждении командой разработчиков.\\
Выбор архитектуры --- один из самых важных пунктов при разработке ПО, поскольку от этого будет зависеть не только процесс разработки, но и работоспособность после релиза. Данное решение невозможно принять разработчику в узком смысле, поскольку не входит в число строго хорошо поставленных задач тимлидом.

\begin{frame} \frametitle{Критерии хорошей архитектуры}
\begin{itemize}
	\item Эффективность системы
	\item Гибкость системы
	\item Расширяемость системы
	\item Масштабируемость процесса разработки
	\item Тестируемость
	\item Возможность повторного использования
\end{itemize}

\end{frame}
\lecturenotes
\\Критерии хорошей архитектуры:
\begin{itemize}
	\item Эффективность системы. В первую очередь программа, конечно же, должна решать поставленные задачи и хорошо выполнять свои функции, причем в различных условиях. Сюда можно отнести такие характеристики, как надежность, безопасность, производительность, способность справляться с увеличением нагрузки (масштабируемость) и т.п.
	\item Гибкость системы. Любое приложение приходится менять со временем — изменяются требования, добавляются новые. Чем быстрее и удобнее можно внести изменения в существующий функционал, чем меньше проблем и ошибок это вызовет — тем гибче и конкурентоспособнее система. 
	\item Расширяемость системы. Возможность добавлять в систему новые сущности и функции, не нарушая ее основной структуры. На начальном этапе в систему имеет смысл закладывать лишь основной и самый необходимый функционал. Но при этом архитектура должна позволять легко наращивать дополнительный функционал по мере необходимости.
	\item Масштабируемость процесса разработки. Возможность сократить срок разработки за счёт добавления к проекту новых людей. Архитектура должна позволять распараллелить процесс разработки, так чтобы множество людей могли работать над программой одновременно.
	\item Тестируемость. Код, который легче тестировать, будет содержать меньше ошибок и надежнее работать.
	\item Возможность повторного использования. Систему желательно проектировать так, чтобы ее фрагменты можно было повторно использовать в других системах.
\end{itemize}

\begin{frame} \frametitle{Архитектурные шаблоны}
	Примеры архитектурных шаблонов: 
	\begin{itemize}
		\item Многоуровневый шаблон (Layered pattern)
		\item Шаблон посредника (Broker pattern)
		\item Шаблон «Модель-Представление-Контроллер» (Model-View-Controller pattern)
		\item Клиент-серверный шаблон (Client-Server pattern)
	\end{itemize}	
\end{frame}
\lecturenotes
\\Для удовлетворения проектируемой системы различным атрибутам качества применяются различные архитектурные шаблоны (паттерны). Каждый шаблон имеет свои задачи и свои недостатки.\\
\begin{itemize}
\item Многоуровневый шаблон (Layered pattern). Система разбивается на уровни, которые на диаграмме изображаются один над другим. Каждый уровень может вызывать только уровень на 1 ниже него. Таким образом разработку каждого уровня можно вести относительно независимо, что повышает модифицируемость системы. Недостатками данного подхода являются усложнение системы и снижение производительности.
\item Шаблон посредника (Broker pattern). Когда в системе присутствует большое количество модулей, их прямое взаимодействие друг с другом становится слишком сложным. Для решения проблемы вводится посредник (например, шина данных), по которой модули общаются друг с другом. Таким образом, повышается функциональная совместимость модулей системы. Все недостатки вытекают из наличия посредника: он понижает производительность, его недоступность может сделать недоступной всю систему, он может стать объектом атак и узким местом системы.
\item Шаблон «Модель-Представление-Контроллер» (Model-View-Controller pattern). Т.к. требования к интерфейсу меняются чаще всего, то возникает потребность часто его модифицировать, при этом сохраняя корректное взаимодействие с данными (чтение, сохранение). Для этого в шаблоне Model-View-Controller (MVC) интерфейс отделён от данных. Это позволяет менять интерфейсы, равно как и создавать их разные варианты.
\item Клиент-серверный шаблон (Client-Server pattern). Если есть ограниченное число ресурсов, к которым требуется ограниченный правами доступ большого числа потребителей, то удобно реализовать клиент-серверную архитектуру. Такой подход повышает масштабируемость и доступность системы. Но при этом сервер может стать узким местом системы, при его недоступности становится недоступна вся система.
\end{itemize}

\subsection{Оценка времени}
\begin{frame} \frametitle{Определение оценки времени разработки ПО}
	\begin{minipage}{0.46\textwidth}
		\begin{block}{В узком смысле}
			Процесс, когда разработчик оценивает сложность задачи с~учетом его сил и~опыта, и~обозначает сколько времени ему понадобится на~её решение
		\end{block}
	\end{minipage}
	\hfill
	\begin{minipage}{0.47\textwidth}
		\begin{block}{В широком смысле}
			Декомпозиция всего процесса разработки на более мелкие задачи, где~итоговое время равно сумме времени, необходимого на~каждую отдельную задачу
		\end{block}{}
	\end{minipage}
\end{frame}
\lecturenotes
\\Поскольку в любом договоре при разработке ПО требуется указать конкретное время выполнения работы, грамотная оценка необходимого времени является очень важными моментом. Неграмотная оценка может привести к нехватке времени и срыванию сроков, что негативно скажется как на итоговой прибыли, так и на репутации и отношении с клиентом. Поэтому разработчику необходимо ответственно подходить к данному вопросу.\\ 
Процесс оценки времени на разработку проекта состоит из декомпозиции общей сложной задачи на подзадачи, необходимое время на которые уточняются у назначенных на них разработчиков, прибавляя к этому затраты на риски или накладные расходы (например: тестирование задачи, тимлидинг, рефакторинг и тд), а затем суммируются и формируют итоговое время. Однако поскольку оценка времени очень неоднозначна и носит очень субъективный характер, она не всегда может совпадать с предполагаемым временем, необходимым для решения задачи. Поэтому многие опытные разработчики, как правило, закладывают в общее время риски возможных проблем, перемножая итоговое время на коэффициент, выведенный ими личным опытом.

\begin{frame} \frametitle{Примеры оценки времени разработки ПО}
	\begin{minipage}{0.47\textwidth}
		\textbf{В узком смысле:}
		
		Обычный рядовой программист получает задание (написание функции, рефакторинг участка кода, написание тестов и т.д.) и~указывает необходимое время на~его выполнение.
	\end{minipage}
	\hfill
	\begin{minipage}{0.5\textwidth}
		\textbf{В широком смысле:}
		
		Тимлид, распределяющий задачи между рядовыми разработчиками, сопостовляет указанное ими время и~делает оценку времени для~функционального блока или~проекта в~целом.
	\end{minipage}
\end{frame}
\lecturenotes
В данных примерах показана стандартная схема работы отдела разработчиков ПО и метода оценки времени. Сперва руководитель или тимлит ставит объемную задачу, которую дробит на более мелкие подзадачи, а затем распределяет их между подчиненными. В его задачу входит не просто распросить сотрудников: "сколько понадобится времени?" и просуммировать, но а так же учесть возможные риски возникающих проблем: ошибки, необходимость рефакторинга и несоблюдение сроков у рядовых разработчиков.\\
Обычно, опытные тимлиды используют специальные формулы рассчета необходимого времени, выделяемого на выполнение задач для рядовых разработчиков: например, берется примерное необходимое время на решение задачи, умножается на определенный коэффициент и суммируется по всем мелким задачам.\\
В свою очередь, обычные сотрудники оценивают время лишь на поставленную им задачу, обговаривая сроки дедлайнов с тимлидом.

\subsection{Обсуждение задач с командой}
\begin{frame} \frametitle{Определение обсуждения задач с~командой}
	
	\begin{minipage}{0.46\textwidth}
		\begin{block}{В узком смысле}
			Процесс обсуждения задачи (возникших проблем, способов решения и~т.д.) с~командой разработчиков или~предоставление отчета о~работе тимлиду
		\end{block}
	\end{minipage}
	\hfill
	\begin{minipage}{0.47\textwidth}
		\begin{block}{В широком смысле}
			Аудит итогов работы рядовых разработчиков с~целью подведения итогов текущей работы и~оптимального распределения дальнейших задач
		\end{block}{}
	\end{minipage}
\end{frame}
\lecturenotes
Разработчик ПО периодически (как правило по окончанию спринта) должен предоставить отчет о проделанной работе руководителю проекта или своему тимлиду.\\
Как правило между спринтами организуются встречи, на которых команда разработчиков подводит итоги прошлого спринта, обсуждает результаты, ставит задачи на будущий спринт и проводит их распределение между разработчиками.\\
Обсуждение задач с командой --- промежуточный процесс при разработке ПО, в котором опытные разработчики рассказывают о состоянии работы на данный момент и выбирают/распределяют дальнейшие действия.\\
Поскольку в процессе разработки большого продукта ПО участвует множество людей, необходимо периодически проводить встречи, на которых оценивается общий прогресс (и/или регресс) по проекту для быстрого выявления багов, неправильной архитектуры и логики, мозговой штурм сложных моментов, обсуждение комментариев заказчика о промежуточных результатах, а также принятия решения о дальнейшей разработке и распределение конкретных задач между разработчиками. 

\begin{frame} \frametitle{Примеры обсуждения задач с~командой}
	\begin{minipage}{0.47\textwidth}
		\textbf{В узком смысле:}
		
		Обычный рядовой программист, столкнувшись с~проблемой при~выполнении своей задачи, обращается за~помощью к~своему~коллеге.
	\end{minipage}
	\hfill
	\begin{minipage}{0.5\textwidth}
		\textbf{В широком смысле:}
		
		Тимлид, прослушав итоги работы своих подчиненных, проводит обсуждение и~ставит дальнейшие задачи для~команды разработчиков.
	\end{minipage}
\end{frame}
\lecturenotes
Столкнувшись с тяжелой, по мнению разботчика, проблемой, он зачастую путается и зацикливается в разработанной им же логике, поэтому в такие моменты не помешает свежий взгляд на данную трудность. С этим могут помочь коллеги занимающиеся на том же проекте, так как они знают большенство его нюансов.\\
Тимлид, по окончанию сроков спринта, проводит собрание разработчиков, на котором поочерено выслушивает отчеты и составляет дальнейший план по работе над проектом.\\

\subsection{Тимлидинг}
\begin{frame} \frametitle{Определение тимлидинга}
	\begin{block}{Тимлидинг в~широком~смысле}
		Процесс контроля за~разработкой ПО. Как правило, этим~занимается один человек --- тимлид. Его задачей является мониторинг и~контроль и~инициация практически всех~процессов разработки, а~так~же непосредственное вмешательство в~какой-либо процесс при~необходимости.
	\end{block}
\end{frame}
\lecturenotes
\\Тимлид — это IT-специалист, который управляет своей командой разработчиков, владеет технической стороной, принимает участие в работе над архитектурой проекта, занимается ревью кода, а также разработкой некоторых особо сложных заданий на проекте.\\
Фактически, team lead — это руководитель команды: команды разработчиков, QA, команды из разномастных специалистов. Его можно условно назвать капитаном, или лидером команды, что ясно из названия. Действительно, в большинстве компаний тимлиды подобны капитанам спортивных команд, так как выполняют схожие обязанности.\\
Team leader, зачастую, на высоком уровне владеет технической стороной, умеет сам заниматься разработкой, тестированием или дизайном. Он может принимать участие в разработке архитектуры проекта, лично заниматься написанием кода, тестированием и т.д.\\

\begin{frame} \frametitle{Примеры тимлидинга}
\begin{itemize}
	\item Планирование 
	\begin{itemize}
		\item Кодревью – чтение готового кода другого разработчика, после выполнения его задачи, с~целью выявления ошибок в~логике
		\item Оценка и~распределение задач между~членами команды 
	\end{itemize}
	\item Контроль над выполнением проекта
	\begin{itemize}
		\item Приём выполненных задач в~команде
		\item Моторинг проблем и~поиск быстрого решения 
		\item Feedback от~разработчиков о~ходе выполнения задач
		\item Взаимодействие с~клиентом 
	\end{itemize}
\end{itemize}
\end{frame}
\lecturenotes
//Планирование. Важной частью работы любой IT-команды является планирование: оценка времени, необходимого для выполнения задачи, а также планирование времени и приоритетов решения задач и подзадач. В процессе планирования разрабатывается план проекта, его схема-архитектура, распределяются задачи между участниками команды.
//Контроль над выполнением проекта. В обязанности Team Lead’a входит обеспечение выполнения проектов в оцененный ранее срок и контроль его соответствия техническим требованиям. С этой целью тимлиды проводят совещания внутри проектной команды, перераспределяют задания, а также занимаются планированием уже в процессе работы над проектом.

\subsection{Тестирование}
\begin{frame} \frametitle{Определение тестирования}
	\begin{minipage}{0.46\textwidth}
		\begin{block}{В узком смысле}
			Тестирование ПО --- запланированный процесс, целью которого является выявление некорректной работы программы или~её~части для~последующего исправления
		\end{block}
	\end{minipage}
	\hfill
	\begin{minipage}{0.5\textwidth}
		\begin{block}{В широком смысле}
			Оценка качества ПО --- техника контроля качества программного продукта, включающая в~себя проектирование тестов, выполнение тестирования и~анализ полученных результатов
		\end{block}
	\end{minipage}
\end{frame}
\lecturenotes
\\Тестирование --- процесс проверки корректной работы программы (или её части) относительно её технического задания при разных вводимых данных.\\
Поскольку ПО должно точно следовать техническому заданию, в задачи разработчика входит проверка соответствия работы заданной спецификации. Данной проблемой обычно занимается специализированный сотрудник – тестировщик. Для тестирования используются разные методы: от визуальной проверки, до написания отдельных программ, симулирующих определенное поведение пользователя.\\
Исходными данными и высшим приоритетом при выборе показателей качества в большинстве случаев являются назначение, функции и функциональная пригодность соответствующего программного средства. Достаточно полное и корректное описание этих свойств должно служить базой для определения значений большинства остальных характеристик и атрибутов качества. Принципиальные и технические возможности и точность измерения значений атрибутов характеристик качества всегда ограничены в соответствии с их содержанием. Это определяет рациональные диапазоны значений каждого атрибута, которые могут быть выбраны на основе здравого смысла, а также путем анализа прецедентов в спецификациях требований реальных проектов.

\begin{frame} \frametitle{Примеры тестирования}
	\begin{minipage}{0.44\textwidth}
		\textbf{В узком смысле:}
		
		Разработчик, закончив писать код функции, проверяет её~работу на~разных вводимых данных.
	\end{minipage}
	\hfill
	\begin{minipage}{0.52\textwidth}
		\textbf{В широком смысле:}
		
		Тимлид, по~окончанию работы над~функциональным блоком, отправляет его ~тестировщикам на~проверку соответствия техническому заданию.
	\end{minipage}
\end{frame}
\\Цели тестирования могут отличаться, в зависимости от этапа разработки ПО, на котором оно проводится. К примеру, на этапе кодирования целью тестирования будет вызов как можно большего количества сбоев в работе программы, что позволит локализовать и исправить дефекты. В то же время, при приемочном тестировании необходимо показать, что система работает правильно. В период сопровождения, тестирование в основном необходимо для того, чтобы убедится в отсутствии новых багов, появившихся во время внесения изменений.\\
Все зависит от организации процесса и его участников. В IT-индустрии большие компании, как правило, имеют команду специалистов, ответственных за оценку соответствия продукта установленным заказчиком требованиям. То есть, отдел качества. Более того, сами разработчики тоже проводят тестирование, которое называется модульным. 


\subsection{Рефакторинг}
\begin{frame} \frametitle{Определение рефакторинга}
	\begin{block}{Рефакторинг это}
		Процесс изменения внутренней структуры программы, не~затрагивающий её~внешнего поведения и~имеющий целью облегчить понимание её~работы. В~основе рефакторинга лежит последовательность небольших эквивалентных (то~есть сохраняющих поведение) преобразований. 
	\end{block}
\end{frame}
\lecturenotes
\\При разработке ПО, иногда возникают ситуации, при которых принимается решение переделать часть кода программы --- произвести рефакторинг. В данном процессе разработчику выделяется участок, который необходимо переписать, продумывается исправленная архитектура и заменяется на новый, улучшенный вариант.\\
В основе рефакторинга лежит последовательность небольших преобразований программного кода, сохраняющих его поведение. Так как каждое преобразование по объёму незначительно, то программисту легче проследить за его правильностью, а вся последовательность этих изменений может привести к существенной перестройке программы и улучшению её согласованности, четкости и простоты понимания её кода другими разработчиками.\\
Рефакторинг изначально не предназначен для исправления ошибок и добавления новой функциональности, он вообще не меняет поведение программного обеспечения и это помогает избежать ошибок и облегчить добавление функциональности. Он выполняется для улучшения понятности кода или изменения его структуры, для удаления «мёртвого кода» — всё это для того, чтобы в будущем код было легче поддерживать и развивать. В частности, добавление в программу нового поведения может оказаться сложным с существующей структурой — в этом случае разработчик может выполнить необходимый рефакторинг, а уже затем добавить новую функциональность.

\begin{frame} \frametitle{Пример рефакторинга}
\textbf{Технический долг} --- это метафора программной инженерии, обозначающая накопленные в~программном коде или~архитектуре проблемы, связанные с~пренебрежением к~качеству при~разработке программного обеспечения и~вызывающие дополнительные затраты труда в~будущем.
 
\textbf{Закрытие технического долга} является типичной причиной рефакторинга. 
\end{frame}
\\При разработке программ постоянно приходится балансировать между скоростью разработки и качеством того, что получается. Чтобы релиз случился вовремя, зачастую приходится жертвовать качеством разработки, используя «костыли» и «хаки». Исправление этого обычно переносят на будущие релизы – что, собственно, и называют «техническим долгом».\\
Технический долг возникает по многим причинам. Заказчики разработки могут больше фокусироваться на сроках и функциональности, нежели на исправлении старых проблем и создании надежного фундамента для дальнейшей разработки.\\
В конечном итоге, технический долг приводит к проблемам у пользователей, которые постепенно прекращают пользоваться продуктом. А игнорирование технического долга как разработчиками, так и менеджерами только способствует его накоплению.


\subsection{Документирование}
\begin{frame} \frametitle{Определение документирования}
	\begin{minipage}{0.45\textwidth}
		\begin{block}{В узком смысле}
			Документирование кода --- это~вставка в~код определенных комментариев, которые позволяют в~дальнейшем упростить работу с~кодом, как~автору, так~и~другим программистам
		\end{block}
	\end{minipage}
	\hfill
	\begin{minipage}{0.5\textwidth}
		\begin{block}{В широком смысле}
			Документирование ПО --- создание печатных руководств пользователя, диалоговой (оперативной) документации и~справочного текста, описывающие, как~пользоваться программным продуктом
		\end{block}
	\end{minipage}
\end{frame}
\lecturenotes
\\В процессе разработки ПО, разработчику всегда необходимо тем или иным образом документировать (комментировать) как промежуточные результаты, так и итоговый. При разработке ПО, разработчику часто приходится сталкиваться с чужим кодом, и, чтобы сэкономить время на изучение функционала, используют комментарии --- документацию.\\
В широком смысле документирование --- это составление разных глобальных инструкций по проекту. Иногда, для создания документации, нанимают особого сотрудника --- технического писателя, он занимается написанием технического документа и пользовательской документации. Дизайн-документом занимается либо аналитик, либо менеджер, API может сгенерировать тимлид.

\subsection{Сопровождение}
\begin{frame} \frametitle{Определение сопровождения ПО}
	\begin{block}{Сопровождение ПО}
		Сопровождение (поддержка) программного обеспечения --- процесс улучшения, оптимизации и~устранения дефектов программного обеспечения (ПО) после~передачи в~эксплуатацию.
	\end{block}
\end{frame}
\lecturenotes
\\Сопровождение ПО — это одна из фаз жизненного цикла программного обеспечения, следующая за фазой передачи ПО в эксплуатацию.\\
В ходе сопровождения в программу вносятся изменения, с тем, чтобы исправить обнаруженные в процессе использования дефекты и недоработки, а также для добавления новой функциональности, с целью повысить удобство использования (юзабилити) и применимость ПО.\\

В процессе сопровождения в программное обеспечение вно сятся следующие изменения, значительно различающиеся при чинами и характеристиками;
\begin{itemize}
\item Исправление ошибок - корректировка программ, выдающих неправильные результаты в условиях, ограниченных техническим заданием и документацией. Исправление ошибок требуют около 20\% общих затрат на сопровождение.
\item Регламентированная документами адаптация программного обеспечения к условиям конкретного использования, с учетом характеристик внешней среды или конфигурации аппаратуры, на которой предстоит функционировать программам. Адаптация занимает около 20\% общих затрат на сопровождение.
\item Модернизация - расширение функциональных возмож ностей или улучшение характеристик решения отдельных задач в соответствии с новым или дополнительным техническим зада нием на программное изделие. Модернизация занимает до 60\% общих затрат на сопровождение.
\end{itemize}

\begin{frame} \frametitle{Типичные задачи сопровождения}
В задачи сопровождения и~продвижения программного обеспечения входит:
\begin{itemize}
	\item Адаптация
	\item Расширение функционала
	\item Корректировка документации
	\item Исправление ошибок
	\item Обучение персонала
\end{itemize}
\end{frame}
\lecturenotes
\\В задачи сопровождения и продвижения программного обеспечения входит:\\
\begin{itemize}
	\item Адаптация программного обеспечения к рабочим условиям. Например, настройка удаленного доступа или адаптация под параметры аппаратуры, на которой будет установлена программа.
	\item Расширение функционала. По мере расширения компании появляется необходимость в новых функциях действующего ПО. Правильно их добавить может только специалист, который занимался разработкой.
	\item Корректировка документации. Любое изменение, которое вносится в уже действующую программу, фиксируется в документации на ПО.
	\item Исправление ошибок. Выявление всех неисправностей возможно только после полноценного ввода ПО в эксплуатацию.
	\item Обучение персонала. После внедрения программного обеспечения, мы обучаем сотрудников компании-заказчика работе со всеми функциями новой системы. Это ускоряет работу и помогает использовать ПО по-максимуму. 
\end{itemize}

\begin{frame} \frametitle{Пример сопровождения ПО}
	После~релиза продукта заказчик потребовал модернизировать продукт: добавить новый функционал. Тимлид декомпозирует поставленную задачу и~распределет среди~команды разработчиков. Рядовые разработчики просто выполняют поставленную задачу: им~не~важно был уже~релиз продукта или~нет.
\end{frame}
\lecturenotes
\\Рассмотрим пример сопровождения ПО: компания-заказчик изначально договорилась о сопровождении их продукта после релиза и требует модернизировать функционал. Поступивший запрос рассматривается тимлидом, который принимает решение о том, как данная задача будет выполнена. После декомпозиции задачи, он выдает задания подчиненным разработчикам, которые, в свою очередь, пишут код, не обязательно зная о том, что продукт уже был выпущен. Именно поэтому спосровождение ПО не рассматривается в узком смысле - для рядового разработчика нет разницы для чего именно пишется код, он просто выполняет свою задачу.


\end{document}

%%% Local Variables: 
%%% mode: TeX-pdf
%%% TeX-master: t
%%% End: 
