\documentclass{../industrial-development}
\graphicspath{{2-software-roles/}}

\title{Роли сотрудников в бизнес-процессах компании по разработке программного обеспечения.}
\author{Гончаров Александр, галки Влад ИТ-21 МО}
\date{}

\begin{document}

\begin{frame}
  \titlepage
\end{frame}

\begin{frame}{План лекции}
  \tableofcontents
\end{frame}


\section{Разработчик }

\begin{frame} \frametitle{Разработчик}
	\begin{block}{}
		\alert {}На самом базовом уровне разработчик – это работник, от которого ждут способности перевести алгоритмы и технические спецификации в код, который может быть исполнен на компьютере. Он должен знать синтаксис языка, однако на этом его знания ограничиваться не должны. Разработчик должен быть способен не только следовать указаниям при обучении новым технологиям и надстройкам, но и быть способен анализировать их, понимать, а также, при возможности, улучшать.
		
	\end{block}
	
\end{frame}

\begin{frame} \frametitle{Разработчик: задачи}
	\begin{itemize}
		\item Использование синтаксиса языка
		\item Анализ и понимание новых технологий
		\item Технологическое консультирование
		\item Проектирование и осуществление реализации
		\item Разработка приложений
		\item Разработка инфраструктуры
	\end{itemize}
\end{frame}

\begin{frame} \frametitle{Разработчик: необходимая подготовка}
	\begin{itemize}
		\item Навыки алгоритмиста 
		\item Знания о синтаксисе языка
		\begin{block}{}
			\alert {Разработчик} должен иметь навыки алгоритмиста и обладать знаниями синтаксиса языка.Как и во многих отраслях, на собеседовании разработчик может столкнуться с работодателем, требующим наличие опыта. Однако это не должно остановить перспективного рекрута, ведь есть множество мест, где он мог бы набраться опыта, таких как некоммерческие организации или конкурсы по программированию. 
		\end{block}
	\end{itemize}
	
\end{frame}

\begin{frame} \frametitle{Разработчик: достоинства роли}
	\begin{itemize}
		\item Карьерный рост
		\item Незаменимость, востребованность 
		\item Возможность выбора конкретной специализации
		\item Невысокие требования по коммуникации с коллегами
		\item Возможность удаленной работы, свободный график
		\item Высокий доход
	\end{itemize}
\end{frame}

\begin{frame} \frametitle{Разработчик: недостатки роли}
	\begin{itemize}
	  \item Большой объем работ
	  \item Зависимость от менеджмента
 	  \item Подходит далеко не всем по психологическим причинам
 	  \item Невозможно получить результат с первого раза
  	  \item Обязателен высокий уровень владения английским
	\end{itemize}
\end{frame}

\section{Тестировщик }

\begin{frame} \frametitle{Тестировщик}
	\begin{block}{}
		\alert {Тестировщик} – роль работника, ответственного за обеспечение определенного уровня качества для финального клиента путём помощи команде разработки в поиске и определении проблем в процессе.
	\end{block}
	
\end{frame}

\begin{frame} \frametitle{Тестировщик: задачи}
	\begin{itemize}
		\item Создание тестовых случаев и скриптов
		\item Отчетность о тестах
		\item Планирование тестов
		\item Случайное тестирование
		\item Выполнение скриптов и наблюдение за результатами 
		\item Повышение надежности кода и продукта
	\end{itemize}
\end{frame}

\begin{frame} \frametitle{Тестировщик: необходимая подготовка}
	\begin{itemize}
		\item Базовое понимание работы процесса
		\item Навыки в составлении документации
		\item Внимание к деталям
		\item Наблюдательность 
	\end{itemize}
	\begin{block}{}
		\alert {}Роль QA – точка входа в сам процесс разработки. Необходимо лишь базовое понимание работы процесса, также приветствуется наличие опыта. 
	\end{block}
	
\end{frame}

\begin{frame} \frametitle{Тестировщик: достоинства роли}
	\begin{itemize}
		\item Высокая вовлеченность в процесс, связь с проектом
		\item Имеет возможность импользовать автоматизацию тестирования
		\item Возможность удаленной работы, свободный график
		\item Высокий доход
	\end{itemize}
\end{frame}

\begin{frame} \frametitle{Тестировщик: достоинства роли}
	\begin{itemize}
		\item Высокая вовлеченность в процесс, связь с проектом
		\item Имеет возможность импользовать автоматизацию тестирования
		\item Возможность удаленной работы, свободный график
		\item Высокий доход
	\end{itemize}
\end{frame}

\begin{frame} \frametitle{Тестировщик: недостатки роли}
	\begin{itemize}
		\item ??? Влад
		\item Необходимость искать граничные случаи
	\end{itemize}
\end{frame}



\section{Инженер по автоматизации тестирования}

\begin{frame} \frametitle{Инженер по автоматизации тестирования}
	\begin{block}{}
		\alert {Инженер по автоматизации тестирования} - роль работника, ответственного за создание автоматических тестов, которые запускаются регулярно при каких-то действиях разработчиков позволяя разработчикам избежать ошибок, которые проверяются в тесте.
	\end{block}
	
\end{frame}

\begin{frame} \frametitle{Инженер по автоматизации тестирования: задачи}
	\begin{block}{}
	\begin{itemize}
		\item Создание автоматических тестов
		\item Планирование эффективного запуска автоматических тестов
		\item Повышение автоматизации рутинных задачь для других участников проекта		
	\end{itemize}
	\end{block}
	
\end{frame}

\begin{frame} \frametitle{Инженер по автоматизации тестирования: достоинства роли}
	\begin{block}{}
		\begin{itemize}
			\item ??? Влад
		\end{itemize}
	\end{block}
	
\end{frame}

\begin{frame} \frametitle{Инженер по автоматизации тестирования: недостатки роли}
	\begin{block}{}
		\begin{itemize}
			\item ??? Влад
		\end{itemize}
	\end{block}
	
\end{frame}

\section{Дизайнер }

\begin{frame} \frametitle{Дизайнер}
	\begin{block}{}
		\alert {Дизайнер} - роль работника, которая отвечает за создание возуальных интерфейсов для пользователей, а также создание макетов по которым разработчики будут создавать проект. Также важно чтобы дизайнер понимал важность правильно спроектированного проекта, чтобы пользователю было удобно использовать проект.
	\end{block}
	
\end{frame}

\begin{frame} \frametitle{Дизайнер: задачи}
	\begin{block}{}
		\begin{itemize}
			\item Создание дизайна приложений
			\item Создание макета приложений для разработчиков
			\item Проектировние дизайна для удобного использования
			\item  ??? Влад
		\end{itemize}
	\end{block}
\end{frame}

\begin{frame} \frametitle{Дизайнер: достоинства роли}
	\begin{block}{}
		\begin{itemize}
			\item Творчество (??? Влад, как опсать?)
			\item  ??? Влад
		\end{itemize}
	\end{block}
\end{frame}

\begin{frame} \frametitle{Дизайнер: недостатки роли}
	\begin{block}{}
		\begin{itemize}
			\item Необходимость проектировать похожие элементы несколько раз
			\item 
			\item  ??? Влад
		\end{itemize}
	\end{block}
\end{frame}


\section{Технический писатель }

\begin{frame} \frametitle{Технический писатель}
	\begin{block}{}
		\alert {}Инфа про Технический писатель
	\end{block}
	
\end{frame}

\section{Бизнес Аналитик }

\begin{frame} \frametitle{Бизнес Аналитик}
	\begin{block}{}
		\alert {}Инфа про Бизнес Аналитик
	\end{block}
	
\end{frame}

\section{Администратор} 

\begin{frame} \frametitle{Администратор}
	\begin{block}{}
		\alert {}Инфа про Администратор
	\end{block}
	
\end{frame}


\end{document}
%%% Local Variables: 
%%% mode: TeX-pdf
%%% TeX-master: t
%%% End: