\documentclass{../industrial-development}
\graphicspath{{14-leadership-and-development-team-management/}}

\title{Лидерство руководителя и управление командой разработчиков}
\author{Грязева А.И. Камуз А.В. ИВТ-21 МО}
\date{}

\begin{document}

\begin{frame}
  \titlepage
\end{frame}

\section{Кто такой Руководитель}


\begin{frame} \frametitle{Понятие руководителя}

\begin{block}{}
Руководитель команды разработчиков --- это IT-специалист, который \textbf{управляет командой} разработчиков, владеет технической стороной, принимает участие в работе над архитектурой проекта, занимается ревью кода, а также разработкой некоторых особо сложных заданий на проекте
\end{block}

\end{frame}



\begin{frame} \frametitle{Качества руководителя}

  \begin{itemize}
  \item \textbf{Ответственность}: для того чтобы не бояться принимать решения и нести личную ответственность за решения 
  \item \textbf{Дипломатичность}: для того чтобы успешно разрешать возникающие конфликты и противоречия. Без этого качества не обойтись в ситуации, когда руководителю необходимо указать разработчику на недостатки в его работе
  \item \textbf{Самосовершенствование}: руководитель должен замечать и исправлять слабые места в своем руководстве
  \item \textbf{Оптимизм}: руководителю необходимо воодушевлять разработчиков на достижение целей
    \end{itemize}
  \end{frame}
\lecturenotes
Ответственность: для того чтобы не бояться принимать решения и нести личную ответственность за решения
 
Дипломатичность: для того чтобы успешно разрешать возникающие конфликты и противоречия. Без этого качества не обойтись в ситуации, когда руководителю необходимо указать разработчику на недостатки в его работе

Самосовершенствование: руководитель должен замечать и исправлять слабые места в своем руководстве
Оптимизм: руководителю необходимо воодушевлять разработчиков на достижение целей
Уверенность: Уверенный руководитель должен быть уравновешен и спокоен. Это положительно влияет на атмосферу работы разработчиков 

Стрессоустойчивость: удерживание команды от излишних эмоций, умение быстро принимать ответственные решения при разных обстоятельствах

Коммуникабельность: способность построить систему коммуникаций в команде, получать надежную информацию и эффективно ее оценивать

Уверенность: уверенный руководитель должен быть уравновешен и спокоен. Это положительно влияет на атмосферу работы разработчиков 

Стрессоустойчивость: удерживание команды от излишних эмоций, умение быстро принимать ответственные решения при разных обстоятельствах

Коммуникабельность: способность построить систему коммуникаций в команде, получать надежную информацию и эффективно ее оценивать

Лидерство: руководитель должен уметь вести за собой сотрудников, мотивировать их на достижение целей

   
\begin{frame} \frametitle{Качества руководителя}

  \begin{itemize}
  
  \item \textbf{Уверенность}: уверенный руководитель должен быть уравновешен и спокоен. Это положительно влияет на атмосферу работы разработчиков 
 \item \textbf{Стрессоустойчивость}: удерживание команды от излишних эмоций, умение быстро принимать ответственные решения при разных обстоятельствах
 \item \textbf{Коммуникабельность}: способность построить систему коммуникаций в команде, получать надежную информацию и эффективно ее оценивать
 \item \textbf{Лидерство}: руководитель должен уметь вести за собой сотрудников, мотивировать их на достижение целей
  \end{itemize}
\end{frame}

\lecturenotes

\section{Кто такой Лидер}

\begin{frame} \frametitle {Понятие лидера и лидерства}

\begin{block}{}
Лидерство --- способность \textbf{влиять} как на отдельную личность, так и на группу, \textbf{направляя} усилия всех на достижение целей компании. Это естественный социально-психологический процесс, основанный на влиянии личного авторитета человека на поведенческий аспект группы
\end{block}


\begin{block}{}
Лидер --- это человек, который может \textbf{влиять} на поведение других людей, \textbf{брать} на себя \textbf{ответственность}, последовательно \textbf{идти к достижению} конкретных \textbf{целей} и \textbf{вести} за собой \textbf{команду}
\end{block}

\end{frame}


\begin{frame} \frametitle{Виды лидерства}
Характеристика стилей лидерства по преобладающим функциям определяет такие виды: 
  \begin{itemize}
  \item Лидер--организатор 
  \item Лидер--творец
  \item Лидер--борец  
  \item Лидер--дипломат
  \item Лидер--утешитель
  \end{itemize}
\end{frame}

\begin{frame} \frametitle{Виды лидерства. Лидер--организатор}

\begin{itemize}
  \item Воспринимает нужды коллектива как собственные 
  \item Оптимистичен, уверен, что большинство проблем разрешимы
  \item Умеет убеждать  
  \item Склонен поощрять, неоодобрение выражает не задевая достоинства коллеги
  \item Результат: команда старается работать лучше
  \end{itemize}

\end{frame}


\begin{frame} \frametitle{Виды лидерства. Лидер--творец}
\begin{itemize}
  \item Способен видеть и внедрять новое
  \item Отсутствует необходимость командовать коллегами -- следование по собственному желанию
  \item Умение заинтересовать команду решением задачи или возникшей проблемы 
  \end{itemize}
\end{frame}

\begin{frame} \frametitle{Виды лидерства. Лидер--борец}
\begin{itemize}
  \item Свойственны воля и уверенность в своих силах
  \item Готов к атаке, отстаивая свои убеждения и мнения коллектива
  \item Бескомпромиссный 
  \item Недостатки: лидер не всегда в состоянии предусмотреть последствия своих действий
  \end{itemize}

\end{frame}

\begin{frame} \frametitle{Виды лидерства. Лидер--дипломат}
\begin{itemize}
  \item Отдает предпочтение доверительным встречам с членами команды
  \item Свойственны мягкие способы влияния
  \item Редко афиширует о планах 
  \item Готов выслушать собеседника
  \end{itemize}
\end{frame}

\begin{frame} \frametitle{Виды лидерства. Лидер--утешитель}
\begin{itemize}
  \item Готов поддержать в трудную минуту
  \item Свойственны вежливость, предупредительность и сопереживание
  \item Почтителен и благожелателен 
  \item Готов выслушать собеседника
  \end{itemize}

\end{frame}



\begin{frame} \frametitle {Основные принципы лидерства для руководителя}

\begin{itemize}
	\item \textbf{Понимание}: определитесь с тем, куда вы идете
	\item \textbf{Передача знаний}: делитесь своими знаниями так, чтобы подчиненные все поняли
	\item \textbf{Делегирование}: общие задачи нужно решать общими усилиями
	\item \textbf{Проверка}: проверяйте свои действия и то, что делают ваши подчиненные в контексте достижения 	поставленных целей
	\item \textbf{Участие}: погружайтесь в работу с головой --- будьте примером для остальных
\end{itemize}
\end{frame}

\begin{frame} \frametitle {Основные принципы лидерства. Блок схема}

\begin{table}[]
\begin{tabular}{lllll}
\cline{3-3}
\multicolumn{1}{c}{} & \multicolumn{1}{l|}{} & \multicolumn{1}{c|}{Понимание}       & --\textgreater{} & \multicolumn{1}{c}{} \\ \cline{3-3}
                     &                       &                                      &                  &                      \\ \cline{3-3}
                     & \multicolumn{1}{l|}{} & \multicolumn{1}{l|}{Передача знаний} & --\textgreater{} &                      \\ \cline{3-3}
                     &                       &                                      &                  &                      \\ \cline{3-3}
                     & \multicolumn{1}{l|}{} & \multicolumn{1}{c|}{Делегирование}   & --\textgreater{} &                      \\ \cline{3-3}
                     &                       &                                      &                  &                      \\ \cline{3-3}
                     & \multicolumn{1}{l|}{} & \multicolumn{1}{c|}{Проверка}        & --\textgreater{} &                      \\ \cline{3-3}
                     &                       &                                      &                  &                      \\ \cline{3-3}
\multicolumn{1}{c}{} & \multicolumn{1}{l|}{} & \multicolumn{1}{c|}{Участие}         &                  &                      \\ \cline{3-3}
\end{tabular}
\end{table}

\end{frame}


\begin{frame} \frametitle {Области деятельности}

Свое развитие основные принципы лидерства получают в следующих областях деятельности:

\begin{itemize}
	\item \textbf{Наставничество}: учите окружающих учить
	\item \textbf{Вознаграждение}: награждая сотрудников за выдающиеся результаты, вы сможете создать ситуацию, в которой одни успехи выливаются в другие
	\item \textbf{Исправление}:помогая сотрудникам учиться на собственных ошибках, повышайте их квалификацию
	\item \textbf{Предвидение}: предугадывайте проблемы, пока они не ударили по вашей команде
	\item \textbf{Адаптация}: совершенствуйтесь, извлекая уроки из собственных ошибок
\end{itemize}
\end{frame}




  
  


\section{Сходства и различия между руководителем и лидером}

\begin{frame} \frametitle{Основные различия руководства и лидерства}
\begin{block}{}
Руководство имеет место в системе формальных (или официальных) отношений
\end{block}

\begin{block}{}
Лидерство --- порождение системы неформальных отношений
\end{block}

\end{frame}

\begin{frame} \frametitle{Основные различия руководства и лидерства}
\begin{block}{}
Руководитель может не быть лидером, тогда эффективность его управления снижается
\end{block}

\begin{block}{}
Лидер может быть руководителем (формальное лидерство), а может им и не являться, и иметь неформальную основу (не формальное лидерство)
\end{block}

\end{frame}


\begin{frame} \frametitle{Основные различия руководства и лидерства}
\begin{block}{}
Руководитель \textbf{определяет как} и какими способами нужно \textbf{достигать} поставленные организацией \textbf{цели}, \textbf{организует} и направляет \textbf{работу членов команды} в соответствии с планами, занимая  пассивную позицию. Строит взаимодействие с окружающими на основе регламентации прав и обязанностей, не выходит за их рамки, \textbf{стремится к порядку и дисциплине}
\end{block}

\begin{block}{}
Лидер реализует функции, ожидаемые коллективом и самостоятельно определяет его цели
\end{block}

\end{frame}

\begin{frame} \frametitle{Основные различия руководства и лидерства}
\begin{block}{}
Руководителю члены команды обязаны подчиняться, за что и получают вознаграждение или наказание
\end{block}

\begin{block}{}
Лидер ведет за собой остальных, а те выступают по отношению к нему не подчиненными, а последователями. Строит отношения с членами команды на основе доверия
\end{block}

\end{frame}


\begin{frame} \frametitle{Основные сходства руководства и лидерства}

\begin{itemize}
	\item \textbf{Руководитель} может быть \textbf{лидером}, также как и \textbf{лидер} может быть \textbf{руководителем}
	\item И руководитель, и лидер имеют власть, хотя характер этой власти разный (личностный и организационный)
	\item И руководитель, и лидер влияют на окружающих, разница этих влияний в целях (личные цели или цели организации) и способах осуществления этого влияния
	
\end{itemize}

\end{frame}



\begin{frame} \frametitle {Лидерство руководителя}

\begin{block}{}
Задача руководителя --- стать не формальным, а подлинным лидером. Это повышает неформальные организационные качества команды, эффективность ее работы. Наиболее удачное сочетание: одновременно руководитель и лидер
\end{block}

\end{frame}




\section{Командообразование (Teambuilding)}

\begin{frame} \frametitle{Понятие тимбилдинга}
\begin{block}{}
Тимбилдинг --- создание благоприятных условий для~работы команды, осуществление мероприятий, нацеленных на сплочение коллектива и его организованности
\end{block}
\end{frame}

\begin{frame} \frametitle{Задачи тимбилдинга}
\begin{itemize}
\item знакомство сотрудников разных отделов между собой
\item создание условий для неформального общения
\item повышение эффективности работы команды
\item повышение уровня взаимодействия между сотрудниками
\item сплочение коллектива

\end{itemize}
\end{frame}

\begin{frame} \frametitle{Задачи тимбилдинга}
\begin{itemize}

\item оценка роли каждого «игрока» в команде: выявление лидеров и аутсайдеров
\item освоение навыков решения нестандартных ситуаций
\item повышение мотивации на достижение коллективных целей
\item снятие стресса, усталости
\item возможность для сотрудников почувствовать себя в новой роли
\end{itemize}
\end{frame}


\begin{frame} \frametitle{Виды тимбилдинга}
\begin{itemize}
\item Экстремальные
\item Интеллектуальные
\item Творческие
\end{itemize}
\end{frame}
\lecturenotes
Экстремальные. Связаны с экстремальными видами спорта и несут риск для здоровья или жизни. Данная группа мероприятий дает практически мгновенный результат и наиболее глубокое чувство единства

Интеллектуальные. Этнические, квесты, реалити шоу, ролевые мероприятия. Главный критерий – умственная работа и наличие смекалки. Отличный способ для участников проявить скрытые таланты и потенциал

Творческие. Решает такие вопросы, как построение отношений в коллективе на основании вкусовых предпочтений и глубокой эмоциональной сплоченности. В отличие от экстремальной группы, эффект достигается не так быстро, но, при регулярном проведении мероприятий – является более прочным и долговременным

\begin{frame} \frametitle{Примеры экстремального тимбилдинга}
\begin{itemize}
\item Приключенческие гонки
\item Ориентирование
\item Сплав на плотах
\end{itemize}
\end{frame}
\lecturenotes
Приключенческие гонки. Организуется тренинг формирования команды по принципу приключенческой гонки, проходящей в экстремальных условиях. Члены команды преодолевают дистанцию, выполняя различные задания и обязательно находя контрольные пункты. Участники не только преодолевают расстояние, но также проявляют свои знания, учась взаимодействовать и обсуждать планируемые действия

Ориентирование. Членам команды нужно с помощью компаса и карт находить необходимое количество пунктов, расположенных на местности. В рамках такого тимбилдинга каждый сотрудник может принимать решения, проявляя свои лидерские качества либо отрабатывая навыки подчиненного, необходимо соответствующее доверие и эффективные коммуникации в команде. Другим распространенным видом спорта становится джип-ориентирование – когда сотрудники на джипах ориентируются по заброшенной местности

Сплав на плотах. Сотрудники на мостах преодолевают водные препятствия, спускаясь по рекам. Здесь команде сотрудников предстоит самостоятельно добывать чистую воду, еду, защищать плот и преодолевать препятствия

\begin{frame} \frametitle{Примеры интеллектупльного тимбилдинга}
\begin{itemize}
\item Квест в городе/Ориентирование в городе
\item Старинные русские ремесла
\item Турнир по ролевым настольным играм
\item Турнир по шахматам
\end{itemize}
\end{frame}
\lecturenotes
Квест в городе / Ориентирование в городе. Организуется определенное состязание, головоломка с интенсивной коммуникацией внутри команды и проявлением творческих талантов

Старинные русские ремесла. Изучение старинных ремесел в вид чеканки, ковки, резьбы по дереву, вышиванию, гончарное ремесло

\begin{frame} \frametitle{Примеры творческого тимбилдинга}
\begin{itemize}
\item Театральные постановки
\item Исторические тимбилдинги
\item Military
\item Литературные тимбилдинги
\item Кулинарные тимбилдинги
\end{itemize}
\end{frame}
\lecturenotes
Театральные постановки. Интересные театральные постановки являются основным звеном формирования команд, получая опыт коллективного взаимодействия. Работники самостоятельно организуют и ставят свой спектакль, а компания привлекает для этого профессионального режиссера

Исторические тимбилдинги. При составлении сценария мероприятия можно использовать различные исторические сюжеты и факты. Отличная возможность получить новые эмоции и яркий опыт. В том числе проведение рыцарских сражений, дней пионеров, походов викингов и пр.

Military. Военная атмосфера, артефакты, заброшенные полигоны – всё это позволит воссоздать идеальные условия для командного взаимодействия. Например, пейнтбол

Литературные тимбилдинги. Корпоратив с использованием литературных сюжетов, создавая отличные условия для командного творчества. Вечер чтения стихов собственного сочинения

Кулинарные тимбилдинги. Возможность проявить свои кулинарные таланты и познакомиться с интересными рецептами. Высокая степень эмоционального сплочения – с готовкой и поеданием интересных блюд. Команда учится взаимодействовать, проявляя свои таланты и лидерские качества в неформальной обстановке


\begin{thebibliography}{99}

%Информацию собирал с нескольких мест на один слайд, поэтому указал списком, а не метками

С. Архипенков Руководство командой разработчиков разработчиков ПО;
Государственный университет  - Высшая школа экономики Факультет Бизнес-информатики Учебное пособие «Лидерство и управление командой»;
Дж. Рейнвотер Как пасти котов 
https://www.kom-dir.ru/article/1127-razvitie-liderskih-kachestv
https://www.kom-dir.ru/article/1466-stili-liderstva
Ицхак Кальдерон Адизес Развитие лидеров. Как понять свой стиль управления и эффективно общаться с носителями иных стилей
\end{thebibliography}

\end{document}

%%% Local Variables: 
%%% mode: TeX-pdf
%%% TeX-master: t
%%% End: 
