\documentclass{../industrial-development}
\graphicspath{{template/}}

\title{Процесс превращения разработчика в~менеджера}
\author{Румянцева Мария Сергеевна, \\Феофанова Алла Александровна, \\ПИ-21 МО}
\date{}

\begin{document}

\begin{frame}
  \titlepage
\end{frame}

\section{Разработчик и менеджер}

\begin{frame} \frametitle{Разработчик и IT-менеджер}
	\textbf{Разработчик} — программист, который занимается созданием различных продуктов в ИТ: компьютерных игр, мобильных приложений, веб-сайтов и др. 	\\
~\\
	\textbf{IT-менеджер} – это специалист, чьей главной задачей является управление проектом в целом: проектирование и расстановка приоритетов, планирование выполнения задач, контроль, коммуникации, а также оперативное решение проблем.
\end{frame}
\lecturenotes
IT-менеджер – это специалист, чьей главной задачей является управление проектом в целом: проектирование и расстановка приоритетов, планирование выполнения задач, контроль, коммуникации, а также оперативное решение проблем.
IT-менеджеры — это сотрудники, отвечающие за качество услуг, предоставляемых пользователям (сотрудникам компании, а иногда — её клиентам и партнёрам) на базе информационных технологий.
IT-менеджер компании – сотрудник, управляющий информационными процессами. Он разбирается не только в технических аспектах IT-среды, но и в вопросах ее взаимодействия с другими сферами: финансовой, кадровой, рыночной.  Задача менеджера внутри фирмы: знать цели развития бизнеса компании, уметь представить бизнес-процессы для их автоматизации, осуществить правильный выбор информационной системы и рассчитать эффект от ее применения.  Его внешние цели – обеспечить клиентам получение качественных ИТ-услуг или организовать продажу информационных продуктов.
Специфика деятельности разработчика (другое название этой профессии — Developer) всецело зависит от выбранного направления. К примеру, разработкой программного обеспечения прикладного характера (игры‚ бухгалтерские программы‚ редакторы‚ мессенджеры, ПО для систем видео- и аудионаблюдения) занимаются прикладные программисты; созданием операционных систем, работой с сетями, написанием интерфейсов к базам данных озадачены системные программисты; воплощением в жизнь проектов веб-дизайнеров, т. е. созданием сайтов, заняты веб-программисты.
~\cite{How_to_be_a_good_IT-manager}


\begin{frame} \frametitle{Важные качества разработчика в IT}
	 \begin{itemize}
	 	\item Сильные технические навыки
		\item Аналитический склад ума
	 	\item Готовность к обучению
	 	\item Умение работать в команде
		\item Стрессоустойчивость
		\item Вовлеченность в рабочий процесс
		\item Способность к концентрации
	 \end{itemize}
\end{frame}
\lecturenotes
Сфера информационных технологий и телекоммуникаций отличается стремительным развитием, и для сохранения набранного темпа становятся всё необходимей специалисты, которые помогли бы ей в этом. В то же время, специалистам следует соответствовать определённым требованиям, без них они не смогут успешно справляться со своими профессиональными задачами. Так какие же личные качества помогают работникам в сфере IT и телекоммуникаций стать настоящими профессионалами?
Такие специалисты в первую очередь должны быть склонны к математике, информатике и работе с техникой, в том числе с компьютерами. Им необходимо иметь аналитический склад ума, хорошую память и способность работать с большим количеством информации. Также незаменимыми качествами для всех сотрудников в этой области, независимо от должности, являются ответственность, организованность, стрессоустойчивость, умение самостоятельно обучаться по специализированной литературе.
Стоит отметить, что облик IT-специалиста, работавшего, к примеру, десять лет назад, существенно отличается от современного. Теперь это не молчаливый, сосредоточенный человек, не отрывающийся от компьютера весь рабочий день, а коммуникабельный сотрудник, готовый к работе в команде и прямому диалогу с клиентами. Создавая продукт, работники IT-сферы и телекоммуникации ориентируются на его будущих потребителей, поэтому хорошо знают интересы и потребности своих потенциальных клиентов.
Важно не просто иметь все эти качества, но и грамотно себя преподносить, например, во время трудоустройства. Главная визитная карточка здесь – резюме, в нём стоит обязательно указать свои преимущества и способности. 
У обладателей таких качеств действительно много шансов найти работу по своей специальности, тем более что спрос на работников в сфере информационных технологий и телекоммуникаций заметно превышает предложение. Конечно, для того чтобы быть компетентными и востребованными им понадобится не только психологическая склонность к профессии. Им не обойтись без хорошего образования, профессиональных знаний и умений, постоянного совершенствования своих навыков. А знание английского языка, к тому же, поможет найти работу в уже известной или очень перспективной зарубежной компании.
~\cite{How_to_be_a_good_IT-manager}



\subsection{Основания для перевода разработчика в менеджера}
\begin{frame} \frametitle{Основания компании для перевода разработчика в менеджеры}
\begin{itemize}
		\item Стремление <<вырастить>> менеджера внутри компании		
		\item Отсутствие менеджера, при невозможности увеличения штата сотрудников
		\item Дефицит времени на введение в работу человека со стороны	
		\item Понимание специфики организации и отрасли
		\item Знание сильных и слабых сторон коллектива
	\end{itemize}
\end{frame}
\lecturenotes
???
~\cite{How_to_be_a_good_IT-manager}

\begin{frame} \frametitle{Необходимые качества IT-менеджера}
\begin{itemize}
		\item Лидерские качества
		\item Умение находить общий язык
		\item Знание процессов разработки и особенностей продукта изнутри
		\item Готовность взять ответственность не только за себя, но и за других
		\item Желание развиваться
		\item Доверие (авторитет) у заказчика
		\item Достижение определенных профессиональных высот

	\end{itemize}
\end{frame}
\lecturenotes
???
~\cite{How_to_be_a_good_IT-manager}

\subsection{Подготовка менеджера проекта}
\begin{frame} \frametitle{Подготовка менеджера проекта}

	 	\begin{table}[H]
\caption{\label{tab:canonsummary} Обучение специалистов}
\begin{center}
\begin{tabular}{|p{0.2\linewidth}|p{0.8\linewidth}|}
\hline
\textbf{Вид} & \textbf{Специфика} \\
\hline
Официаль-ные &  обучение специалистов в ВУЗ-ах и на специальных курсах, завершение которых удостоверяется соответствующим документом. \\
\hline
Полуофи-циальные  & прохождение краткосрочных курсов, посещение популярных лекций и практических занятий. \\
\hline
Неофициаль-ные &  участие в конференциях, семинарах и ознакомление с соответствующей литературой \\
\hline
Обучение в процессе работы & обучение на рабочем месте, а также самообразование. \\
\hline
\end{tabular}
\end{center}
\end{table} 
\end{frame}
\lecturenotes
официальные – обучение специалистов в ВУЗ-ах и на специальных курсах, завершение которых удостоверяется соответствующим документом. К слушателям этой формы подготовки предъявляются обязательные требования, как при начале, так и при завершении обучения, а в ряде случаев, и по ходу самого обучения;
полуофициальные – прохождение насыщенной программы обучения на краткосрочных курсах (продолжительностью от нескольких дней до нескольких месяцев), посещение популярных лекций и практических занятий. К слушателям этой формы подготовки не предъявляются какие-либо обязательные требования, и они не получают специальных удостоверений об их окончании;
неофициальные – участие в конференциях, симпозиумах, региональных семинарах, собраниях профессиональных обществ, а также ознакомление с соответствующей литературой;
обучение в процессе работы – это обучение на рабочем месте при выполнении конкретного проекта, а также самообразование.
~\cite{How_to_be_a_good_IT-manager}

\subsection{Отличия характера работы и стиля мышления}
\begin{frame} \frametitle{Отличия характера работы разработчика и менеджера}
	\begin{block}{Разработчик}
\begin{itemize}
\item Отсутсвтует общение с заказчиком
\item Должен уметь читать чужой код и разбираться в нем
\item Направленность на решение конкретной задачи
\item Ответственен только за себя
\end{itemize}
\end{block}
\end{frame}
\lecturenotes

~\cite{How_to_be_a_good_IT-manager}

\begin{frame} \frametitle{Отличия характера работы разработчика и менеджера}
	\begin{block}{IT-менеджер}
\begin{itemize}
\item Общение с заказчиком
\item Постоянное взаимодействие с командой
\item Правильное распределение бюджета и времени сотрудников
\item Повышенная ответственность (за проект)
\end{itemize}
\end{block}
\end{frame}
\lecturenotes

~\cite{How_to_be_a_good_IT-manager}

\begin{frame} \frametitle{Отличия стиля мышления разработчика и менеджера}
	\begin{block}{Разработчик}
\begin{itemize}
\item Узкое мышление
\item Согласование изменений с teamleader и IT-менеджером
\item Глубокое погружение в свой процесс работы
\end{itemize}
\end{block}
\end{frame}
\lecturenotes

~\cite{How_to_be_a_good_IT-manager}

\begin{frame} \frametitle{Отличия стиля мышления разработчика и менеджера}
	\begin{block}{IT-менеджер}
\begin{itemize}
\item Видимость проекта в целом
\item Согласование изменений с заказчиком 
\item Мотивирует команду на результат
\end{itemize}
\end{block}
\end{frame}
\lecturenotes

~\cite{How_to_be_a_good_IT-manager}

\begin{frame} \frametitle{Типы программистов (по мнению Дж. Ханка Рейнвотера)}
	\begin{block}{Распространенные породы:}
\begin{itemize}
\item Архитектор
\item Реконструктивист 
\item Художник
\item Инженер
\item Ученый
\item Лихач
\end{itemize}
\end{block}
\end{frame}
\lecturenotes

~\cite{How_to_be_a_good_IT-manager}

\begin{frame} \frametitle{Типы программистов (по мнению Дж. Ханка Рейнвотера)}
	\begin{block}{Редкие породы:}
\begin{itemize}
\item Волшебник
\item Минималист
\item Аналогист
\item Трюкач
\end{itemize}
\end{block}
\end{frame}
\lecturenotes

~\cite{How_to_be_a_good_IT-manager}

\begin{frame} \frametitle{Типы программистов (по мнению Дж.~Ханка Рейнвотера)}
	\begin{block}{Дворовые породы:}
\begin{itemize}
\item Разгильдяй
\item Тормоз
\item Любитель
\item Профан
\item Эклектик
\end{itemize}
\end{block}
\end{frame}
\lecturenotes

~\cite{How_to_be_a_good_IT-manager}


\begin{frame} \frametitle{Негативные эталоны в менеджменте (по мнению Дж.~Ханка Рейнвотера)}
	\begin{enumerate}
\item Руководители типа <<Мелочная опека>>:
		 \begin{itemize}
                     \item Всезнайка
 		 \item Диктатор
 		 \item Генерал
		\end{itemize} 
\item Неорганизованные руководители:	
		 \begin{itemize}
                     \item Скарлетт О'Хара
		 \item Временщик
 		\item Новичок
		\end{itemize} 
\item Гений
\item Строитель империй тьмы
\end{enumerate}
\end{frame}
\lecturenotes

~\cite{How_to_be_a_good_IT-manager}

\begin{frame} \frametitle{Преимущества менеджера с техническим бэкграундом}
	 \begin{itemize}
                      \item Понимание технической стороны работы
		\item Общение с командой на одном языке
		\item Реальная оценка трудоемкости задач
		\item Способность убедить заказчика в корректности поставленных менеджером сроков и задач
		\item Грамотное распределение задач в команде (особенно в отсутствие teamleader)
	\end{itemize} 	
\end{frame}
\lecturenotes

~\cite{How_to_be_a_good_IT-manager}

\begin{frame} \frametitle{Недостатки менеджера с техническим бэкграундом}
	 \begin{itemize}
                      \item Оценивание сроков задач исходя из своего опыта без учета мнения команды
		\item Недостаточное понимание экономической стороны вопроса
		\item Выполнение чужих обязанностей при попытке помочь команде
		\item Недопонимание заказчика
		\item Неспособность посмотреть на проект в целом
	\end{itemize} 	
\end{frame}
\lecturenotes
Менеджер сосредотачивается на исполнительской части, видит отдельные задачи\nobreakdash-«кирпичики», но недостаточно ясно оценивает перспективы
Создаваемый в ходе проекта, глазами потребителей, а для хорошего менеджера проекта это важно \nobreakdash- видеть пользовательские характеристики продукта, понимать, как их можно улучшить.
Есть риск отказа от хороших идей, потому что менеджер считает их реализацию слишком сложной, долгой и дорогой. В то время как менеджер-не разработчик мог бы ухватиться за эту идею, и другие участники команды нашли бы эффективный путь ее воплощения в жизнь.
~\cite{How_to_be_a_good_IT-manager}

\begin{frame} \frametitle{Преимущества менеджер с экономическим бэкграундом}
	 \begin{itemize}
                      \item Грамотная оценка стоимости задач
		\item Общение с заказчиком на его языке
		\item Предоставление свободы действий (самостоятельности) в команде
		\item Видение целостного результата проекта
	\end{itemize} 	
\end{frame}
\lecturenotes

~\cite{How_to_be_a_good_IT-manager}

\begin{frame} \frametitle{Недостатки менеджера с экономическим бэкграундом}
	 \begin{itemize}
                      \item Неправильная оценка реальной трудоемкости задачи и постановки сроков
		\item Перекладывание ответственности на плечи teamleader и команды
		\item Неграмотное распределение задач между членами команды
		\item Отсутствие технических знаний не позволяет увидеть риски проекта
	\end{itemize} 	
\end{frame}
\lecturenotes

~\cite{How_to_be_a_good_IT-manager}



\begin{frame} \frametitle{Ошибки начинающего менеджера}
\begin{itemize}	
		\item Совмещение функций менеджера и разработчика	
		\item Стремление выполнить всю работу самостоятельно
	 	\item Узкое мышление
	 	\item Уход от ответственности за работу всей команды
		\item Чрезмерный контроль над работой подчиненных
	 	\item Неправильное распределение обязанностей
	 	\item Постановка нереальных сроков для выполнения задач
 \end{itemize}
\end{frame}
\lecturenotes

~\cite{How_to_be_a_good_IT-manager}

\begin{frame} \frametitle{Ценность менеджера для компании}
 \begin{block}{Успешный менеджер:}
\begin{itemize}
  \item Укладывается в сроки
  \item Учитывает интересы не только заказчика, но и подчиненных 
  \item Способствует личностному росту команды
  \item Обладает большим количеством успешных проектов
  \item Имеет положительные отзывы от клиентов
  \item Поощряет и вознаграждает сотрудников
  \item Может предугадать возможное развитие событий в проекте

  \end{itemize}
 \end{block}
\end{frame}
\lecturenotes


~\cite{How_to_be_a_good_IT-manager}

\section{Организация рабочего процесса IT-менеджера}

\subsection{Задачи IT-менеджера}
\begin{frame} \frametitle{Задачи IT-менеджера}
\begin{itemize}	
		\item Управление персоналом
		\item Расчет бюджета информационной среды
	 	\item Анализ удовлетворенности клиентов
	 	\item Разработка и обеспечение качественных IT-услуг
		\item Переговоры с клиентами и поставщиками
	 	\item Поиск и оценка инновационных технологий, предложение их внедрения
	 	\item Определение сроков работы над IT-проектами
	           \item Реализация IT-стратегии компании
		\item Организация взаимодействия IT-специалистов с другими работниками
 \end{itemize}
\end{frame}
\lecturenotes

~\cite{How_to_be_a_good_IT-manager}

\subsection{Риски в процессе проектирования}
\begin{frame} \frametitle{Риски в процессе проектирования}
 \begin{block}{Риск №~1}
\large{Риск ухудшения позиций компании на рынке}
  \end{block}

 \begin{block}{Причина}
Возникает, когда в существующий продукт вследствие конкуренции приходится вносить новые непредусмотренные изначально черты.
  \end{block}

\end{frame}
\lecturenotes

~\cite{How_to_be_a_good_IT-manager}


\begin{frame} \frametitle{Риски в процессе проектирования}
 \begin{block}{Риск №~2}
\large{Риск неумеренного усложнения процесса}
  \end{block}

 \begin{block}{Причина}
Возникает вследствие жесткой взаимозависимости компонентов-подсистем или негибкости их конфигурации.
  \end{block}

\end{frame}
\lecturenotes

~\cite{How_to_be_a_good_IT-manager}

\begin{frame} \frametitle{Риски в процессе проектирования}
 \begin{block}{Риск №~3}
\large{Риск неоправданной сложности продукта на верхних уровнях архитектуры}
  \end{block}

 \begin{block}{Причина}
Чрезвычайно усложняется задача разработчиков, не принимавших участия в процессе создания системы, но вынужденных выискивать способы исправления базовых компонентов.
  \end{block}

\end{frame}
\lecturenotes

~\cite{How_to_be_a_good_IT-manager}


\begin{thebibliography}{99}
\bibitem{How_to_be_a_good_IT-manager} \href{http://www.pvsm.ru/upravlenie-proektami/36476}
\bibitem{Managers_in_IT} \href{https://habrahabr.ru/post/219741/}
\bibitem{Managers_thinking_style} \href{http://www.hr-asteri.ru/employer/poleznaya_informaciya/stil_myshleniya_i_povedeniya_professionalnogo_rukovoditelya/}
\bibitem{Programmer_vs_manager} \href{https://dou.ua/lenta/articles/programmer-vs-manager/}
\bibitem{Best_qualities_for_IT-manager} \href{http://hr-portal.ru/article/kakie-kachestva-nuzhny-menedzheru-it-proektov}
\bibitem{From_engineer_to_manager_keeping_your_technical_skills} \href{https://hackernoon.com/from-engineer-to-manager-keeping-your-technical-skills-40579cc8ea00}
\bibitem{From_programmer_to_manager} \href{https://m.dotdev.co/what-i-learned-transitioning-from-being-a-programmer-to-an-it-manager-8e58e7b406}
\end{thebibliography}

\end{document}

%%% Local Variables: 
%%% mode: TeX-pdf
%%% TeX-master: t
%%% End: 
