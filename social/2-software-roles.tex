\documentclass{../industrial-development}
\graphicspath{{2-software-roles/}}

\title{Роли сотрудников в бизнес-процессах компании по разработке программного обеспечения.}
\author{Гончаров Александр, Галкин Влад ИТ-21 МО}
\date{}

\begin{document}
	
	\begin{frame}
		\titlepage
	\end{frame}
	
	\begin{frame}{План лекции}
		\tableofcontents
	\end{frame}
	
	
	\section{Разработчик }
	
	\begin{frame} \frametitle{Разработчик}
		\begin{block}{}
			\alert {}На самом базовом уровне разработчик – это работник, от которого ждут способности перевести алгоритмы и технические спецификации в код, который может быть исполнен на компьютере. Он должен знать синтаксис языка, однако на этом его знания ограничиваться не должны. Разработчик должен быть способен не только следовать указаниям при обучении новым технологиям и надстройкам, но и быть способен анализировать их, понимать, а также, при возможности, улучшать.
			
		\end{block}
		
	\end{frame}
	
	\begin{frame} \frametitle{Разработчик: задачи}
		\begin{itemize}
			\item Использование синтаксиса языка
			\item Анализ и понимание новых технологий
			\item Технологическое консультирование
			\item Проектирование и осуществление реализации
			\item Разработка приложений
			\item Разработка инфраструктуры
		\end{itemize}
	\end{frame}
	
	\begin{frame} \frametitle{Разработчик: необходимая подготовка}
		\begin{itemize}
			\item Навыки алгоритмиста 
			\item Знания о синтаксисе языка
			\begin{block}{}
				\alert {Разработчик} должен иметь навыки алгоритмиста и обладать знаниями синтаксиса языка.Как и во многих отраслях, на собеседовании разработчик может столкнуться с работодателем, требующим наличие опыта. Однако это не должно остановить перспективного рекрута, ведь есть множество мест, где он мог бы набраться опыта, таких как некоммерческие организации или конкурсы по программированию. 
			\end{block}
		\end{itemize}
		
	\end{frame}
	
	\begin{frame} \frametitle{Разработчик: достоинства роли}
		\begin{itemize}
			\item Карьерный рост
			\item Незаменимость, востребованность 
			\item Возможность выбора конкретной специализации
			\item Невысокие требования по коммуникации с коллегами
			\item Возможность удаленной работы, свободный график
			\item Высокий доход
		\end{itemize}
	\end{frame}
	
	\begin{frame} \frametitle{Разработчик: недостатки роли}
		\begin{itemize}
			\item Большой объем работ
			\item Зависимость от менеджмента
			\item Подходит далеко не всем по психологическим причинам
			\item Невозможно получить результат с первого раза
			\item Обязателен высокий уровень владения английским
		\end{itemize}
	\end{frame}
	
	\section{Тестировщик }
	
	\begin{frame} \frametitle{Тестировщик}
		\begin{block}{}
			\alert {Тестировщик} – роль работника, ответственного за обеспечение определенного уровня качества для финального клиента путём помощи команде разработки в поиске и определении проблем в процессе.
		\end{block}
		
	\end{frame}
	
	\begin{frame} \frametitle{Тестировщик: задачи}
		\begin{itemize}
			\item Создание тестовых случаев и скриптов
			\item Отчетность о тестах
			\item Планирование тестов
			\item Случайное тестирование
			\item Выполнение скриптов и наблюдение за результатами 
			\item Повышение надежности кода и продукта
		\end{itemize}
	\end{frame}
	
	\begin{frame} \frametitle{Тестировщик: необходимая подготовка}
		\begin{itemize}
			\item Базовое понимание работы процесса
			\item Навыки в составлении документации
			\item Внимание к деталям
			\item Наблюдательность 
		\end{itemize}
		\begin{block}{}
			\alert {}Роль QA – точка входа в сам процесс разработки. Необходимо лишь базовое понимание работы процесса, также приветствуется наличие опыта. 
		\end{block}
		
	\end{frame}
	
	\begin{frame} \frametitle{Тестировщик: достоинства роли}
		\begin{itemize}
			\item Высокая вовлеченность в процесс, связь с проектом
			\item Имеет возможность импользовать автоматизацию тестирования
			\item Возможность удаленной работы, свободный график
			\item Высокий доход
		\end{itemize}
	\end{frame}
	
	\begin{frame} \frametitle{Тестировщик: достоинства роли}
		\begin{itemize}
			\item Высокая вовлеченность в процесс, связь с проектом
			\item Имеет возможность импользовать автоматизацию тестирования
			\item Возможность удаленной работы, свободный график
			\item Высокий доход
		\end{itemize}
	\end{frame}
	
	\begin{frame} \frametitle{Тестировщик: недостатки роли}
		\begin{itemize}
			\item Рутинная проверка кругов функционала
			\item Высокая ответственность выпускающих тестировщиков, обратно пропорциональна мотивации
			\item Большой объем работы
			\item Необходимость искать граничные случаи
		\end{itemize}
	\end{frame}
	
	
	
	\section{Инженер по автоматизации тестирования}
	
	\begin{frame} \frametitle{Инженер по автоматизации тестирования}
		\begin{block}{}
			\alert {Инженер по автоматизации тестирования} - роль работника, ответственного за создание автоматических тестов, которые запускаются регулярно при каких-то действиях разработчиков позволяя разработчикам избежать ошибок, которые проверяются в тесте.
		\end{block}
		
	\end{frame}
	
	\begin{frame} \frametitle{Инженер по автоматизации тестирования: задачи}
		\begin{block}{}
			\begin{itemize}
				\item Создание автоматических тестов
				\item Планирование эффективного запуска автоматических тестов
				\item Повышение автоматизации рутинных задачь для других участников проекта		
			\end{itemize}
		\end{block}
		
	\end{frame}
	
	\begin{frame} \frametitle{Инженер по автоматизации тестирования: достоинства роли}
		\begin{block}{}
			\begin{itemize}
				\item Высокая значимость при выполнении проектов
				\item Изучение новых технологий
			\end{itemize}
		\end{block}
		
	\end{frame}
	
	\begin{frame} \frametitle{Инженер по автоматизации тестирования: недостатки роли}
		\begin{block}{}
			\begin{itemize}
				\item Большой объем кода для покрытия тестами
				\item Преимущественно рутинная работа
			\end{itemize}
		\end{block}
		
	\end{frame}
	
	\section{Дизайнер }
	
	\begin{frame} \frametitle{Дизайнер}
		\begin{block}{}
			\alert {Дизайнер} - роль работника, которая отвечает за создание возуальных интерфейсов для пользователей, а также создание макетов по которым разработчики будут создавать проект. Также важно чтобы дизайнер понимал важность правильно спроектированного проекта, чтобы пользователю было удобно использовать проект.
		\end{block}
		
	\end{frame}
	
	\begin{frame} \frametitle{Дизайнер: задачи}
		\begin{block}{}
			\begin{itemize}
				\item Разработка новых прикладных решений для иконок, шрифтов и т.д.
				\item Создание макета приложений для разработчиков
				\item Проектирование пользовательских интерфейсов
				\item Определение организованности элементов интерфейса;
				\item Группировка элементов интерфейса;
				\item Выравнивание элементов интерфейса;
				\item Создание единого стиля элементов интерфейса;
				\item Разграничение информационных блоков.
			\end{itemize}
		\end{block}
	\end{frame}
	
	\begin{frame} \frametitle{Дизайнер: достоинства роли}
		\begin{block}{}
			\begin{itemize}
				\item Творческая работа
				\item Просторы для креатива
				\item Ответственность за оформление продукта, который увидит конечный пользователь
			\end{itemize}
		\end{block}
	\end{frame}
	
	\begin{frame} \frametitle{Дизайнер: недостатки роли}
		\begin{block}{}
			\begin{itemize}
				\item Необходимость проектировать похожие элементы несколько раз
				\item Непосредственный контакт с заказчиков
			\end{itemize}
		\end{block}
	\end{frame}
	
	
	\section{Технический писатель }
	
	\begin{frame} \frametitle{Технический писатель}
		\begin{block}{}
			\alert {Технический писатель} - специалист, занимающийся документированием в рамках решения технических задач, в частности разработки программного обеспечения
		\end{block}
	\end{frame}

	\begin{frame} \frametitle{Технический писатель: задачи}
		\begin{block}{}
			\begin{itemize}
				\item Оформление документов в соответствии со стандартом (принятым в организации, международным или, например, ГОСТОМ)
				\item Поддерживание документов в актуальном состоянии.
				\item Анализ новых IT-продуктов и программ, а также их тестирование. В конечном счете, подобные занятия помогут специалисту лучше разобраться в новом продукте, а значит написать понятную и подробную инструкцию.
				\item  Сбор информации о продукте
			\end{itemize}
		\end{block}
	\end{frame}

	\begin{frame} \frametitle{Технический писатель:  достоинства роли}
		\begin{block}{}
			\begin{itemize}
				\item Обеспечение необходимой тех. документацией разработку
			\end{itemize}
		\end{block}
	\end{frame}

	\begin{frame} \frametitle{Технический писатель: Недостатки роли}
		\begin{block}{}
			\begin{itemize}
				\item Рутинная работа
			\end{itemize}
		\end{block}
	\end{frame}
	
	\section{Бизнес Аналитик}
	
	\begin{frame} \frametitle{Бизнес Аналитик}
		\begin{block}{}
			\alert {}В классическом понимании, {бизнес-аналитик} — это человек, который анализирует бизнес-потребности организации, а также формулирует пути и схемы усовершенствования бизнес-процессов, осуществляет стратегическое планирование. 
		\end{block}
		
	\end{frame}
	
	\begin{frame} \frametitle{Бизнес-аналитик: задачи}
		\begin{itemize}
			\item Усовершенствование продуктов компании
			\item Работа с клиентом
			\item Контроль над качеством продукта и соответствием требованиям клиента
			\item Формулирование высокоуровневых требований к программному продукту
			\item Составление его структуры и связей между элементами
			\item Определение технологий и/или используемых программных решений
			\item Проектирование юзер-интерфейса, формата и способа взаимодействия между пользователем и программой
		\end{itemize}
	\end{frame}
	
	\begin{frame} \frametitle{Бизнес-аналитик: достоинства роли}
		\begin{itemize}
			\item Полное представление о проекте
			\item Определяющая развитие проекта деятельность
			\item Профессиональные качества можно перенести на другой род  деятельности
			\item Общение с заказчиком
			\item Изучение больших объемов информации в сжатые сроки
		\end{itemize}
	\end{frame}
	
	\begin{frame} \frametitle{Бизнес-аналитик: недостатки роли}
		\begin{itemize}
			\item Большая ответственность и в следствии - стресс
			\item Необходимость постоянного обучения новым подходам и методологиям
			\item Ведение паралельно большого количества проектов
		\end{itemize}
	\end{frame}
	
	
	\section{Администратор} 
	
	\begin{frame} \frametitle{Администратор}
		\begin{block}{}
			\alert {Администратор} - сотрудник, должностные обязанности которого подразумевают обеспечение штатной работы парка компьютерной техники, сети и программного обеспечения. Зачастую системному администратору вменяется обеспечение информационной безопасности в организации.
		\end{block}
	\end{frame}

	\begin{frame} \frametitle{Администратор: задачи}
		\begin{itemize}
			\item Подготовка и сохранение резервных копий данных, их периодическая проверка и уничтожение;
			\item Установка и конфигурирование необходимых обновлений для операционной системы и используемых программ;
			\item Установка и конфигурирование нового аппаратного и программного обеспечения;
			\item Создание и поддержание в актуальном состоянии пользовательских учётных записей;
			\item Ответственность за информационную безопасность в компании;
			\item Устранение неполадок в системе;
			\item Планирование и проведение работ по расширению сетевой структуры предприятия;
			\item Документирование всех произведенных действий.
		\end{itemize}
	\end{frame}

	\begin{frame} \frametitle{Администратор: достоинства роли}
		\begin{itemize}
			\item Большая значимость
			\item Свободный график
		\end{itemize}
	\end{frame}

	\begin{frame} \frametitle{Администратор: недостатки роли}
		\begin{itemize}
			\item Необходимость коммуникации с начальством для обеспечения сети необходимыми устройствами(маршрутизаторы и т.д.)
		\end{itemize}
	\end{frame}

	
	
\end{document}
%%% Local Variables: 
%%% mode: TeX-pdf
%%% TeX-master: t
%%% End:
