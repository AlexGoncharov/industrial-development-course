\documentclass{../industrial-development}
\graphicspath{{2-software-roles/}}

\title{Роли сотрудников в бизнес-процессах компании по разработке программного обеспечения}
\author{Гончаров Александр, Галкин Влад ИТ-21 МО}
\date{}

\begin{document}
	
	\begin{frame}
		\titlepage
	\end{frame}
	
	\begin{frame}{План лекции}
		\tableofcontents
	\end{frame}
	
	
	\section{Разработчик }
	
	\begin{frame} \frametitle{Разработчик}
		\begin{block}{}
			\alert {}Разработчик – работник, от которого ждут способности перевести алгоритмы и технические спецификации в код, который может быть исполнен на компьютере. Он должен знать синтаксис языка, а также должен быть способен не только следовать указаниям при обучении новым технологиям и надстройкам, но и быть способен анализировать их, понимать, и при возможности, улучшать.
			
		\end{block}
		
	\end{frame}
	
	\begin{frame} \frametitle{Разработчик: задачи}
		\begin{itemize}
			\item Использование синтаксиса языка
			\item Анализ и понимание новых технологий
			\item Технологическое консультирование
			\item Реализация продукта
			\item Разработка инфраструктуры
		\end{itemize}
	\end{frame}

	\begin{frame} \frametitle{Разработчик: вовлеченность}
		Разработчик ответственнен за создание кода проекта, и его поддержку. Играет центральную роль в разработке проекта.
	\end{frame}
	
	\section{Тестировщик }
	
	\begin{frame} \frametitle{Тестировщик}
		\begin{block}{}
			\alert {Тестировщик} – роль работника, ответственного за обеспечение определенного уровня качества для финального клиента путём помощи команде разработки в поиске и определении проблем в процессе.
		\end{block}
		
	\end{frame}
	
	\begin{frame} \frametitle{Тестировщик: задачи}
		\begin{itemize}
			\item Создание тестовых случаев и скриптов
			\item Отчетность о тестах
			\item Планирование тестов
			\item Случайное тестирование
			\item Выполнение скриптов и наблюдение за результатами 
			\item Повышение надежности кода и продукта
		\end{itemize}
	\end{frame}
	
	\begin{frame} \frametitle{Тестировщик: необходимая подготовка}
		\begin{itemize}
			\item Базовое понимание работы процесса
			\item Навыки в составлении документации
			\item Внимание к деталям
			\item Наблюдательность 
		\end{itemize}
		\begin{block}{}
			\alert {}Роль QA – точка входа в сам процесс разработки. Необходимо лишь базовое понимание работы процесса, также приветствуется наличие опыта. 
		\end{block}
		
	\end{frame}
		
	\section{Инженер по автоматизации тестирования}
	
	\begin{frame} \frametitle{Инженер по автоматизации тестирования}
		\begin{block}{}
			\alert {Инженер по автоматизации тестирования} - роль работника, ответственного за создание автоматических тестов, которые запускаются регулярно при каких-то действиях разработчиков позволяя разработчикам избежать ошибок, которые проверяются в тесте.
		\end{block}
		
	\end{frame}
	
	\begin{frame} \frametitle{Инженер по автоматизации тестирования: задачи}
			\begin{itemize}
				\item Создание автоматических тестов
				\item Планирование эффективного запуска автоматических тестов
				\item Повышение автоматизации рутинных задачь для других участников проекта		
			\end{itemize}
	\end{frame}
	
	\section{Дизайнер }
	
	\begin{frame} \frametitle{Дизайнер}
		\begin{block}{}
			\alert {Дизайнер} - роль работника, которая отвечает за создание возуальных интерфейсов для пользователей, а также создание макетов по которым разработчики будут создавать проект. Также важно чтобы дизайнер понимал важность правильно спроектированного проекта, чтобы пользователю было удобно использовать проект.
		\end{block}
		
	\end{frame}
	
	\begin{frame} \frametitle{Дизайнер: задачи}
			\begin{itemize}
				\item Разработка новых прикладных решений для иконок, шрифтов и т.д.
				\item Создание макета приложений для разработчиков
				\item Проектирование пользовательских интерфейсов
				\item Определение организованности элементов интерфейса;
				\item Группировка элементов интерфейса;
				\item Выравнивание элементов интерфейса;
				\item Создание единого стиля элементов интерфейса;
				\item Разграничение информационных блоков.
			\end{itemize}
	\end{frame}
	
	
	\section{Технический писатель }
	
	\begin{frame} \frametitle{Технический писатель}
		\begin{block}{}
			\alert {Технический писатель} - специалист, занимающийся документированием в рамках решения технических задач, в частности разработки программного обеспечения
		\end{block}
	\end{frame}

	\begin{frame} \frametitle{Технический писатель: задачи}
			\begin{itemize}
				\item Оформление документов в соответствии со стандартом (принятым в организации, международным или, например, ГОСТОМ)
				\item Поддерживание документов в актуальном состоянии.
				\item Анализ новых IT-продуктов и программ, а также их тестирование. В конечном счете, подобные занятия помогут специалисту лучше разобраться в новом продукте, а значит написать понятную и подробную инструкцию.
				\item  Сбор информации о продукте
			\end{itemize}
	\end{frame}
	
	\section{Бизнес-аналитик}
	
	\begin{frame} \frametitle{Бизнес-аналитик}
		\begin{block}{}
			\alert {}В классическом понимании, {бизнес-аналитик} — это человек, который анализирует бизнес-потребности организации, а также формулирует пути и схемы усовершенствования бизнес-процессов, осуществляет стратегическое планирование. 
		\end{block}
		
	\end{frame}
	
	\begin{frame} \frametitle{Бизнес-аналитик: задачи}
		\begin{itemize}
			\item Усовершенствование продуктов компании
			\item Работа с клиентом
			\item Контроль над качеством продукта и соответствием требованиям клиента
			\item Формулирование высокоуровневых требований к программному продукту
			\item Составление его структуры и связей между элементами
			\item Определение технологий и/или используемых программных решений
			\item Проектирование юзер-интерфейса, формата и способа взаимодействия между пользователем и программой
		\end{itemize}
	\end{frame}
	
	
	\section{Администратор} 
	
	\begin{frame} \frametitle{Системный администратор}
		\begin{block}{}
			\alert {Системный администратор} - сотрудник, должностные обязанности которого подразумевают обеспечение штатной работы парка компьютерной техники, сети и программного обеспечения. Зачастую системному администратору вменяется обеспечение информационной безопасности в организации.
		\end{block}
	\end{frame}

	\begin{frame} \frametitle{Системный администратор: задачи}
		\begin{itemize}
			\item Подготовка и сохранение резервных копий данных, их периодическая проверка и уничтожение;
			\item Установка и конфигурирование необходимых обновлений для операционной системы и используемых программ;
			\item Установка и конфигурирование нового аппаратного и программного обеспечения;
			\item Создание и поддержание в актуальном состоянии пользовательских учётных записей;
			\item Ответственность за информационную безопасность в компании;
			\item Устранение неполадок в системе;
			\item Планирование и проведение работ по расширению сетевой структуры предприятия;
			\item Документирование всех произведенных действий.
		\end{itemize}
	\end{frame}
	
\end{document}
%%% Local Variables: 
%%% mode: TeX-pdf
%%% TeX-master: t
%%% End:
