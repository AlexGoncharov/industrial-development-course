\documentclass{../industrial-development}
\graphicspath{{10-quality-assurance/}}

\title{Обеспечение качества ПО}
\author{Антипова Наталья Андреевна, ПИ-21 МО}
\date{}

\begin{document}

\begin{frame}
  \titlepage
\end{frame}



\section{Проблема качества программного обеспечения}
\begin{frame} \frametitle{Обеспечение качества программного обеспечения}
  \begin{block}{}
	\alert{Обеспечение качества программного обеспечения} (англ. Software quality assurance, SQA) "--- набор процедур мониторинга разработки программного обеспечения и методов, используемых для обеспечения его качества 
  \end{block}
 	 \begin{itemize}
\item В настоящее время большое количество методов SQA соответствуют стандарту ISO 9000
\item Процедуры обеспечения качества охватывают весь цикл разработки программного обеспечения
  	\end{itemize}
\end{frame}

\lecturenotes

	\alert{Обеспечение качества программного обеспечения} (англ. Software quality assurance, SQA) "--- набор процедур мониторинга разработки программного обеспечения и методов, используемых для обеспечения его качества 
	 \begin{itemize}
\item В настоящее время в программной инженерии имеется большое количество методов обеспечения качества программного обеспечения, которые соответствуют одному или нескольким стандартам, в частности стандарту ISO 9000
\item Процедуры обеспечения качества охватывают весь цикл разработки программного обеспечения, включая такие процессы как определение требований, проектирование, кодирование, контроль исходного кода, анализ кода, конфигурационное управление, тестирование, управление релизами и интеграция продуктов
  	\end{itemize}



\begin{frame} \frametitle{Уникальность SQA}
Различия между ПО и другими промышленными продуктами: 
 	 \begin{enumerate}
\item Сложность продукта (огромное количество возможностей ПО)
\item Видимость продукта (программные продукты невидимы)
\item Разработка и производство продукции
	\begin{itemize}
\item Разработка продукции (тестирование прототипа и~обнаружение дефектов)
\item Планирование производства продукции (дополнительная возможность для проверки и~испытания продукта)
\item Производство (процедуры контроля качества для~обнаружения сбоев)
	\end{itemize}
  	\end{enumerate}
\end{frame}

\lecturenotes

Ни один разработчик не заявит, что его программное обеспечение не имеет дефектов, поскольку даже крупные производители компьютерной техники не делают этого

Этот отказ фактически отражает существенные элементарные различия между программным обеспечением и другими промышленными продуктами (автомобилями, стиральными машинами или радиоприемниками). Эти различия можно классифицировать следующим образом:
 	 \begin{enumerate}
\item Сложность продукта. Сложность продукта может быть измерена по количеству режимов работы, разрешенных в нем. Промышленный продукт, даже передовая машина, не допускает более нескольких тысяч режимов работы, созданных комбинациями его различных настроек. В свою очередь, глядя на типичный программный пакет, можно найти миллионы возможностей работы программного обеспечения. Обеспечение правильного определения и разработки множества операционных возможностей является серьезной проблемой для индустрии программного обеспечения
\item Видимость продукта. В то время как промышленные продукты видны, программные продукты невидимы. Большинство дефектов промышленного продукта можно обнаружить в процессе производства. Более того, отсутствие части в промышленном продукте, как правило, весьма заметно (представьте себе дверь, отсутствующую у вашего нового автомобиля). Однако дефекты программных продуктов (хранящихся на дискетах или компакт-дисках) невидимы, так же как и тот факт, что части программного пакета могут отсутствовать с самого начала
\item Разработка и производство продукции. Этапы, на которых может возникнуть возможность обнаружения дефектов промышленного продукта:
	\begin{itemize}
\item Разработка продукции. На этом этапе сотрудники дизайнеров и контроля качества (QA) проверяют и тестируют прототип продукта, чтобы обнаружить его дефекты
\item Планирование производства продукции. На этом этапе производственный процесс и инструменты разрабатываются и готовятся. В некоторых продуктах существует необходимость в специальной производственной линии для проектирования и строительства. Таким образом, эта фаза предоставляет дополнительные возможности для проверки продукта, что может выявить дефекты, которые «избежали» проверок и испытаний, проведенных на этапе разработки
\item Производство. На этом этапе применяются процедуры контроля качества для обнаружения сбоев самих продуктов. Дефекты в изделии, обнаруженные в первый период производства, обычно могут быть исправлены изменением конструкции изделия или материалов, или в производственных инструментах, таким образом, чтобы устранить такие дефекты в изделиях, производимых в будущем~\cite[с.~4--5]{SQA-Galin}
	\end{itemize}
  	\end{enumerate}


\begin{frame} \frametitle{Основные характеристики среды~SQA}
 	 \begin{enumerate}
\item Контрактные условия
\item Поддержка отношений между клиентом и~поставщиком
\item Обязательная работа в команде
\item Сотрудничество и координация с другими командами по разработке ПО
\item Интерфейсы с другими программными системами
\item Необходимо продолжить выполнение проекта, несмотря на изменения количества членов команды
\item Необходимость продолжения технического обслуживания ПО в течение продолжительного периода
  	\end{enumerate}
\end{frame}

\lecturenotes

 	 \begin{enumerate}
\item Контрактные условия. В результате обязательств и условий, определенных в контракте между разработчиком ПО и клиентом, деятельность по разработке и сопровождению ПО должна справляться с:
		 \begin{itemize}
	\item Определенным списоком функциональных требований, которые необходимо выполнить при разработке ПО и его обслуживания
	\item Бюджетом проекта
	\item Графиком проекта
		\end{itemize}
\item Поддержка отношений между клиентом и поставщиком. На протяжении всего процесса разработка и сопровождение программного обеспечения, деятельность находится под контролем клиента. Команда проекта должна постоянно сотрудничать с клиентом: рассмотреть его просьбу о внесении изменений, обсудить его критические замечания по различным аспектам проекта и получить одобрение изменений, инициированных командой разработчиков. 
\item Обязательная работа в команде. Три фактора, как правило, мотивируют создание проектной команды, а не присвоение проекта одному профессионалу:
		 \begin{itemize}
	\item Требования к графику. Другими словами, рабочая нагрузка, выполняемая в течение проектного периода, требует участия более одного человека, если проект должен быть завершен вовремя
	\item Необходимость в различных специализациях для выполнения проекта
	\item Желание воспользоваться профессиональной взаимной поддержкой и обзором для повышения качества проекта
		\end{itemize}
\item Сотрудничество и координация с другими командами по разработке ПО. Выполнение проектов, особенно крупномасштабных проектов, более чем одной командой "--- очень распространенное событие в индустрии программного обеспечения. В этих случаях может потребоваться сотрудничество с:
		 \begin{itemize}
	\item Другими командами разработчиков программного обеспечения в той же организации
	\item Командами разработки оборудования в той же организации
	\item Группами разработчиков программного и аппаратного обеспечения других поставщиков
	\item Командой разработчиков программного обеспечения и оборудования для разработки ПО, которые принимают участие в разработке проекта
		\end{itemize}
\item Интерфейсы с другими программными системами. В настоящее время большинство программных систем включают в себя интерфейсы с другими пакетами программного обеспечения. Эти интерфейсы позволяют передавать данные в электронной форме между программными системами, без ввода данных, обрабатываемых другими программными системами. Можно выделить следующие основные типы интерфейсов:
		 \begin{itemize}
	\item Входные интерфейсы, в которые другие программные системы передают данные в вашу программную систему
	\item Выходные интерфейсы, из которых ваша программная система передает обработанные данные в другие программные системы
	\item Входные и выходные интерфейсы к плате управления 
		\end{itemize}
\item  Необходимо продолжить выполнение проекта, несмотря на изменения количества членов команды. Это весьма характерно для членов команды, чтобы покинуть команду в период развития проекта, либо из-за продвижение на более высокие должности, либо переводы в другой город, и так далее. Затем руководитель группы должен заменить уходящего члена команды другим сотрудником или вновь набранным сотрудником. Независимо от того, сколько усилий вложено в обучение нового члена команды, разработка проекта должна продолжаться
\item Необходимость продолжения технического обслуживания программного обеспечения в течение продолжительного периода. Клиенты, которые разрабатывают или покупают программное обеспечение, рассчитывают продолжать использовать его в течение длительного периода времени, как правило, в течение 5-10 лет. В течение периода обслуживания в конечном счете возникнет необходимость в обслуживании. В большинстве случаев разработчик обязан предоставлять эти услуги~\cite[с.~7--10]{SQA-Galin}
  	\end{enumerate}


\section{Особенности SQA при разработке ПО}
\begin{frame} \frametitle{Причины ошибок в ПО}
 	 \begin{itemize}
\item Неправильное определение требований
\item Нарушение взаимопонимания между клиентом и~разработчиком
\item Преднамеренные отклонения от требований к ПО
\item Логические ошибки проектирования
\item Ошибки программирования
\item Несоблюдение документации и инструкций по~программированию
\item Недостаток процесса тестирования
\item Ошибки операций
\item Ошибки документации
  	\end{itemize}
\end{frame}

\lecturenotes

 	 \begin{itemize}
\item Неправильное определение требований, обычно подготовленное клиентом, является одной из основных причин ошибки программного обеспечения. Наиболее распространенными ошибками этого типа являются:
		\begin{itemize}
	\item Ошибочное определение требований
	\item Отсутствие важных требований
	\item Неполное определение требований
	\item Включение ненужных требований, функций, которые, как ожидается, не будут необходимы в ближайшем будущем
  		\end{itemize}
\item Нарушение взаимопонимания между клиентом и разработчиком. Недоразумения, возникающие в результате некачественного взаимодействия клиента и разработчика, являются дополнительными причинами ошибок, преобладающих на ранних этапах процесса разработки
\item Преднамеренные отклонения от требований к ПО. В нескольких случаях разработчики могут сознательно отклоняться от требований документа, действиЙ, которые часто вызывают ошибки ПО. Ошибки в этих случаях являются побочным продуктом изменений
\item Логические ошибки проектирования. Ошибки программного обеспечения могут войти в систему, когда профессионалы (проектирующие системные архитекторы, программисты, аналитики и т.д.) формулируют требования к программному обеспечению
\item Ошибки программирования. Широкий спектр причин заставляет программистов совершать ошибки программирования. К ним относятся недопонимание конструкторской документации, языковые ошибки на языках программирования, ошибки в применении CASE и другие инструменты разработки, ошибки в выборе данных и т.д.
\item Несоблюдение документации и инструкций по программированию. Почти у каждой единицы разработки есть своя документация и стандарты программирования, которые определяют содержание, порядок и формат документов, а также код, созданный членами команды. Для поддержки этого требования модуль разрабатывает и публикует свои шаблоны и инструкции по программированию. Члены команды разработчиков или подразделения должны соблюдать эти требования
\item Недостаток процесса тестирования. Недостатки процесса тестирования влияют на частоту ошибок, оставляя большее количество ошибок незамеченными или исправленными. Эти недостатки обусловлены следующими причинами:
		\begin{itemize}
	\item Неполные планы тестирования оставляют необработанные части программного обеспечения или прикладных функций и состояния системы
	\item Не удается документировать и сообщать обнаруженные ошибки и ошибки
	\item Недостаточность; оперативное исправление выявленных ошибок программного обеспечения, что является результатом неправильного указания причин отказа
	\item Неполная коррекция обнаруженных ошибок из-за халатности или давления во времени
  		\end{itemize}
\item Ошибки операций. Операции направляют пользователя в отношении действий, необходимых на каждом этапе процесса. Они имеют особое значение в сложных программных системах, где обработка выполняется в несколько этапов, каждая из которых может подавать разнообразные типы данных и позволяет рассматривать результаты промежуточных результатов
\item Ошибки документации. Ошибки документации, которые беспокоят команды разработчиков и разработчиков, представляют собой ошибки в проектных документах и в документации, интегрированной в состав программного обеспечения. Эти ошибки могут вызывать дополнительные ошибки на последующих этапах разработки и во время обслуживания~\cite[с.~19--23]{SQA-Galin}
  	\end{itemize}



\begin{frame} \frametitle{Отличие обеспечение качества ПО от контроля качества}
 	 \begin{itemize}
\item Контроль качества определяется как набор мероприятий, предназначенных для~оценки качества разрабатываемого или~изготовленного продукта
\item Основная цель обеспечения качества "--- минимизировать стоимость гарантированного качества за~счет различных видов деятельности, выполняемых на~всех этапах разработки и~производства
  	\end{itemize}
Отличие: деятельность по контролю качества является лишь частью всего спектра мероприятий по обеспечению качества
\end{frame}


\lecturenotes

\begin{itemize}
\item Контроль качества определяется как «набор мероприятий, предназначенных для оценки качества разрабатываемого или изготовленного продукта» (IEEE, 1991); другими словами, деятельность, основной целью которой является удержание любого продукта, не отвечающего требованиям. Соответственно, проверка контроля качества и другие виды деятельности осуществляются по мере завершения разработки или изготовления продукта до того, как продукт будет отправлен клиенту
\item Основная цель обеспечения качества "--- минимизировать стоимость гарантированного качества за счет различных видов деятельности, выполняемых на всех этапах разработки и производства. Эти действия предотвращают причины ошибок и обнаруживают и исправляют их на раннем этапе процесса разработки. В результате мероприятия по обеспечению качества существенно снижают темпы выпускаемой продукции, которые не имеют права на отгрузку, и в то же время в большинстве случаев снижают затраты на обеспечение качества
  	\end{itemize}
В результате:
 	\begin{enumerate}
\item Деятельность по контролю качества и обеспечению качества выполняет различные задачи
\item Деятельность по контролю качества является лишь частью всего спектра мероприятий по обеспечению качества
  	\end{enumerate}



\begin{frame} \frametitle{Цели SQA при разработке ПО}
 	 \begin{enumerate}
\item ПО дожно соответствовать функциональным техническим требованиям
\item ПО дожно соответствовать управленческим планам и~бюджетным требованиям
\item Инициирование и управление деятельностью по~совершенствованию и~повышению эффективности разработки ПО и~деятельности SQA (улучшение перспектив достижения требований при~снижении затрат на~разработку ПО)
  	\end{enumerate}
\end{frame}

\lecturenotes

 	 \begin{enumerate}
\item Обеспечение приемлемого уровня уверенности в том, что программное обеспечение будет соответствовать функциональным техническим требованиям
\item Обеспечение приемлемого уровня уверенности в том, что программное обеспечение будет соответствовать управленческим планам и бюджетным требованиям
\item Инициирование и управление деятельностью по совершенствованию и повышению эффективности разработки программного обеспечения и деятельности SQA. Это означает улучшение перспектив достижения функциональных и управленческих требований при снижении затрат на разработку программного обеспечения и деятельность в области SQA~\cite[с.~29]{SQA-Galin}
  	\end{enumerate}


\section{SQA в процессе разработки}
\begin{frame} \frametitle{Факторы качества: факторы работоспособности}
 	 \begin{itemize}
\item Корректность (Correctness)
\item Надежность (Reliability)
\item Эффективность (Efficiency)
\item Целостность (Integrity)
\item Удобство использования (Usability)
  	\end{itemize}
\end{frame}

\lecturenotes

 	 \begin{itemize}
\item Корректность (Correctness). Требования к корректности определяются в перечне требуемых выходных данных программной системы, таких как отображение запроса баланса клиента в информационной системе учета продаж или подача воздуха в зависимости от температуры, заданной микропрограммой промышленного блока управления
\item Надежность (Reliability). Требования к надежности касаются отказа в предоставлении услуг. Они определяют максимально допустимую частоту отказов программной системы и могут относиться ко всей системе или к одной или нескольким ее отдельным функциям
\item Эффективность (Efficiency). Требования к эффективности касаются аппаратных ресурсов, необходимых для выполнения всех функций программной системы в соответствии со всеми другими требованиями. Основными аппаратными ресурсами, которые необходимо учитывать, являются возможности обработки компьютера, его емкость хранения данных с точки зрения объема памяти и емкости диска и возможности передачи данных в коммуникационных линиях. Требования могут включать в себя максимальные значения, при которых аппаратные ресурсы будут применяться в разработанной программной системе или прошивке
\item Целостность (Integrity). Требования к целостности касаются безопасности программного обеспечения, то есть требований для предотвращения доступа неавторизованным лицам, для различения между большинством персонала, которому разрешено видеть информацию («разрешение на чтение»), и ограниченной группе, которой будет разрешено добавлять и изменить данные («разрешение на запись») и т.д.
\item Удобство использования (Usability). Требования в отношении удобства использования касаются объема кадровых ресурсов, необходимых для обучения нового сотрудника и работы с программной системой~\cite[с.~38--41]{SQA-Galin}
  	\end{itemize}



\begin{frame} \frametitle{Факторы качества: факторы проверки}
 	 \begin{itemize}
\item Работоспособность (Maintainability)
\item Гибкость (Flexibility)
\item Тестируемость (Testability)
  	\end{itemize}
\end{frame}

\lecturenotes

 	 \begin{itemize}
\item Работоспособность (Maintainability). Требования по обеспечению работоспособности определяют усилия, которые потребуются пользователям и обслуживающему персоналу для выявления причин сбоев в ПО, устранения сбоев и проверки успеха исправлений. Требования этого фактора относятся к модульной структуре ПО, внутренней программной документации и руководству программиста
\item Гибкость (Flexibility). Возможности и усилия, необходимые для поддержки адаптивной поддержки, покрываются требованиями гибкости. К ним относятся ресурсы (т.е. в человеко-днях), необходимые для адаптации пакета программного обеспечения к различным клиентам одной и той же торговли, различной степени активности, различных диапазонов продуктов и т.д. Требования этого фактора также поддерживают совершенную деятельность по обслуживанию, такую как изменения и дополнения к ПО, чтобы улучшить его обслуживание и адаптировать его к изменениям в технической или коммерческой среде фирмы
\item Тестируемость (Testability). Требования к тестированию касаются тестирования информационной системы, а также ее работы. Требования к тестированию для удобства тестирования связаны со специальными функциями программ, которые помогают тестеру, например, путем предоставления предопределенных промежуточных результатов и файлов журналов. Требования к тестируемости, связанные с эксплуатацией программного обеспечения, включают автоматическую диагностику, выполняемую программной системой перед запуском системы, чтобы выяснить, находятся ли все компоненты программной системы в рабочем состоянии и получить отчет об обнаруженных ошибках. Другой тип этих требований касается автоматических диагностических проверок, применяемых техническими специалистами по техническому обслуживанию для выявления причин сбоев программного обеспечения~\cite[с.~41--42]{SQA-Galin}
  	\end{itemize}



\begin{frame} \frametitle{Факторы качества: факторы преобразования}
 	 \begin{itemize}
\item Переносимость (Portability)
\item Повторное использование (Reusability)
\item Совместимость (Interoperability)
  	\end{itemize}
\end{frame}

\lecturenotes

 	 \begin{itemize}
\item Переносимость (Portability). Требования к переносимости имеют тенденцию адаптировать программную систему к другим средам, состоящим из различного оборудования, различных операционных систем и т.д. Эти требования позволяют продолжать использовать одно и то же базовое программное обеспечение в различных ситуациях или использовать его одновременно в различных аппаратных и операционных системах
\item Повторное использование (Reusability). Требования повторного использования касаются использования программных модулей, первоначально разработанных для одного проекта, в новом разрабатываемом программном проекте. Они также могут позволить будущим проектам использовать данный модуль или группу модулей разрабатываемого в настоящее время ПО. Ожидается, что повторное использование ПО сэкономит ресурсы для развития, сократит период разработки и предоставит модули более высокого качества. Эти преимущества более высокого качества основаны на предположении, что большинство ошибок ПО уже были обнаружены благодаря действиям по обеспечению качества, выполняемым на исходном ПО, пользователями исходного ПО и во время его предыдущих повторных использований
\item Совместимость (Interoperability). Требования к совместимости сосредоточены на создании интерфейсов с другими системами программного обеспечения или с прошивкой другого оборудования. Требования к совместимости могут указывать имя (имена) программного обеспечения или прошивки, для которых требуется интерфейс. Они также могут указать структуру вывода, принятую в качестве стандарта, в конкретной отрасли или области приложений~\cite[с.~43--44]{SQA-Galin}
  	\end{itemize}



\begin{frame} \frametitle{Компоненты системы SQA}
 	 \begin{enumerate}
\item Предпроектные компоненты
\item Компоненты оценки жизненного цикла проекта
 		 \begin{itemize}
	\item Компоненты этапа жизненного цикла разработки (отзывы, экспертные заключения, тестирование ПО)
	\item Компоненты этапа обслуживания
  		\end{itemize}
\item Компоненты предотвращения и улучшения ошибок инфраструктуры
\item Компоненты управления качеством программного обеспечения
\item Компоненты оценки стандартизации, сертификации и~SQA
\item Компоненты организационной базы SQA (персонал)
  	\end{enumerate}
\end{frame}

\lecturenotes

Система SQA всегда сочетает в себе широкий спектр компонентов SQA, все из которых используются для опроса множества источников ошибок программного обеспечения и достижения приемлемого уровня качества программного обеспечения. Задача SQA уникальна в области обеспечения качества из-за специфических характеристик программного обеспечения. Кроме того, среда, в которой осуществляется разработка и сопровождение программного обеспечения, напрямую влияет на компоненты SQA

Компоненты системы SQA можно разделить на шесть классов:
 	 \begin{enumerate}
\item Предпроектные компоненты. Обеспечивают, чтобы надлежащим образом были определены проектные обязательства с учетом требуемых ресурсов, графика и бюджета, а также планы развития и качества
\item Компоненты оценки жизненного цикла проекта. Жизненный цикл проекта состоит из двух этапов: этап жизненного цикла разработки и этап обслуживания.
 		 \begin{enumerate}
	\item Компоненты этапа жизненного цикла разработки обнаруживают ошибки проектирования и программирования. Его компоненты разделены на следующие три подкласса: отзывы, экспертные заключения, тестирование программного обеспечения. Компоненты SQA, используемые на этапе эксплуатации, включают в себя специализированные компоненты обслуживания, а также компоненты жизненного цикла разработки, которые применяются в основном для функциональности, улучшающей задачи обслуживания
	\item Дополнительный подкласс классов жизненного цикла проекта SQA связан с обеспечением качества частей проекта, выполняемых субподрядчиками и другими внешними участниками при разработке и обслуживании проекта
  		\end{enumerate}
\item Компоненты предотвращения и улучшения ошибок инфраструктуры. Основными задачами этих компонентов, которые применяются во всей организации, являются устранение или, по крайней мере, снижение частоты ошибок на основе накопленного опыта SQA организации
\item Компоненты управления качеством программного обеспечения. Этот класс компонентов ориентирован на несколько целей, основными из которых являются контроль за разработкой и поддержанием деятельности, а также введение ранних управленческих мер поддержки, которые в основном предотвращают или сводят к минимуму расписание и бюджетные сбои и их результаты
\item Компоненты оценки стандартизации, сертификации и SQA. Эти компоненты внедряют международные профессиональные и управленческие стандарты в организации. Основными задачами этого класса являются унификация международных профессиональных знаний, улучшение координации организационных систем качества с другими организациями и оценка достижений систем качества в соответствии с общим масштабом. Различные стандарты могут быть разделены на две основные группы: стандарты управления качеством и стандарты проектных процессов
\item Компоненты организационной базы SQA (персонал). Организационная база SQA включает менеджеров, персонал тестирования, подразделение SQA и практиков, заинтересованных в качестве программного обеспечения (опекуны SQA, члены комитета SQA и участники форума SQA). Все эти субъекты способствуют качеству программного обеспечения; их основные цели - инициировать и поддерживать внедрение компонентов SQA, выявлять отклонения от процедур SQA и методологию и предлагать улучшения~\cite[с.~57--58]{SQA-Galin}
  	\end{enumerate}



\begin{frame} \frametitle{План обеспечения качества ПО}
 	 \begin{enumerate}
\item Перечень целей обеспечения качества
\item Плановая обзорная деятельность
\item Плановые программные тесты
\item Тесты для программного обеспечения, разработанные извне
\item Инструменты и процедуры управления конфигурацией
  	\end{enumerate}
\end{frame}

\lecturenotes

 	 \begin{enumerate}
\item Перечень целей обеспечения качества. Термин «цели качества» относится к основным требованиям к качеству разрабатываемой системы программного обеспечения. Количественные меры обычно предпочтительнее качественных мер при выборе целей качества, поскольку они обеспечивают разработчику более объективную оценку производительности программного обеспечения в процессе разработки и тестирования системы. Однако, один тип цели не полностью совпадают с другими
\item  Плановая обзорная деятельность. В плане качества должен быть представлен полный перечень всех запланированных мероприятий по обзору: проектные обзоры (DR), проектные проверки, проверки кода и т.д. Для каждого из которых определены следующие:
	 	 \begin{itemize}
	\item Объем деятельности по обзору
	\item Тип рассматриваемой деятельности
	\item Расписание мероприятий по обзору
	\item Специальные процедуры, которые должны применяться
	\item Кто несет ответственность за проведение обзора?
	  	\end{itemize}
\item Плановые программные тесты. В плане качества должен быть представлен полный список запланированных тестов программного обеспечения, причем для каждого теста должны быть указаны следующие:
	 	 \begin{itemize}
	\item Устройство, интеграция или полная система, подлежащая испытанию
	\item Тип проверочных мероприятий, которые должны быть выполнены, включая спецификацию компьютеризированных тестов программного обеспечения, которые будут применяться
	\item Планируемый график испытаний
	\item Специальные процедуры, которые должны применяться
	\item Кто отвечает за проведение теста
	  	\end{itemize}
\item Тесты для программного обеспечения, разработанные извне. Полный комплект испытаний, запланированных для программного обеспечения, разработанного извне, должен предоставляться в рамках плана качества. Включаемые предметы - это приобретенное программное обеспечение, программное обеспечение, разработанное субподрядчиками, и программное обеспечение, поставляемое клиентом. Испытания для программного обеспечения, разработанного извне, должны быть параллельными тем, которые используются для внутренних программных тестов
\item Инструменты и процедуры управления конфигурацией. В плане качества должны быть указаны инструменты и процедуры управления конфигурацией, включая те процедуры контроля изменений, которые должны применяться во всем проекте~\cite[с.~101--103]{SQA-Galin}
  	\end{enumerate}



\begin{frame} \frametitle{Основные инструменты поддержки SQA}
 	 \begin{enumerate}
\item Процедуры технического обслуживания и~рабочие инструкции
\item Обучение и сертификация команд технического обслуживания
\item Устройства, обеспечивающие качество
\item Профилактические и корректирующие действия
\item Управление конфигурацией
\item Документация по техническому обслуживанию и~записи о~качестве
  	\end{enumerate}
\end{frame}

\lecturenotes

Источник~\cite[с.~274]{SQA-Galin}



\begin{frame} \frametitle{Факторы, влияющие на интенсивность SQA}
Факторы проекта:
 	 \begin{itemize}
\item\small Масштабы проекта
\item\small Техническая сложность и трудность
\item\small Степень использования многоразовых программных компонентов
\item\small Серьёзность результатов отказа в~случае сбоя проекта
  	\end{itemize}
Командные факторы:
 	 \begin{itemize}
\item\small Профессиональная квалификация членов команды
\item\small Знакомство команды с проектом и~его опытом в~этой области
\item\small Наличие сотрудников, способных профессионально поддержать команду
\item\small Процент новых сотрудников в~команде~
  	\end{itemize}
\end{frame}
 


\begin{frame} \frametitle{Аспекты качества: верификация}
 	 \begin{itemize}
\item Позволяет доказать, что результат разработки соответствует предъявленным к нему требованиям
\item Выполняется методом проверки характеристик продукции с заданными требованиями, результатом является вывод о соответствии (или несоответствии) продукции
\item Включает в себя инспекции, тестирование кода, анализ результатов тестирования, формирование и анализ отчетов о проблемах
  	\end{itemize}
\end{frame}
 
\lecturenotes 

Verification


\begin{frame} \frametitle{Аспекты качества: валидация}
 	 \begin{itemize}
\item Представляет интересы клиента путем изучения степени выполнения первоначальных требований клиента
\item Позволяет доказать, что в результате разработки были достинуты цели, которые планировалось достичь 
\item Проводится всегда с участием представителей заказчиков, пользователей, бизнес-аналитиков или~экспертов в предметной области 
  	\end{itemize}
\end{frame}
  
\lecturenotes 

Validation
 

\begin{frame} \frametitle{Аспекты качества: квалификация}
 	 \begin{itemize}
\item Фокусируется на эксплуатационных аспектах, где техническое обслуживание является основным вопросом
\item Рассматривает применение в проекте профессиональных стандартов и процедур кодирования, исходя из предположения о том, что применение этих стандартов облегчает техническое обслуживание 
  	\end{itemize}
\end{frame}

\lecturenotes

Qualification. Планировщики деятельности по обеспечению качества должны определять, какой из этих аспектов следует рассматривать в каждом из запланированных мероприятии по обеспечению качества~\cite[с.~133]{SQA-Galin}
 


\begin{frame} \frametitle{Тестирование ПО}
	\begin{block}{}
	\alert{Тестирование программного обеспечения} "--- процесс исследования, испытания программного продукта, имеющий своей целью проверку соответствия между реальным поведением программы и~её ожидаемым поведением на~конечном наборе тестов, выбранных определенным образом
	\end{block}
 	 \begin{itemize}
\item Все связанные тесты проводятся в~соответствии с~утвержденными процедурами тестирования на~одобренных тестовых примерах
\item Тестирования является составной частью процесса верификации
  	\end{itemize}
 \end{frame}

\lecturenotes

 	 \begin{itemize}
\item Все связанные тесты проводятся в~соответствии с~утвержденными процедурами тестирования на~одобренных тестовых примерах
\item Тестирования является составной частью процесса верификации
\item Тестирование стало первым инструментом SQA перед~отправкой или~установкой программного продукта в~помещении заказчика
  	\end{itemize}


\begin{frame} \frametitle{Цели тестирования ПО}
Приямые цели:
 	 \begin{itemize}
\item Определить и выявить как можно больше ошибок в~тестируемом ПО
\item Довести проверенное ПО, после исправления выявленных ошибок и~повторного тестирования, до~приемлемого уровня качества
\item Эффективно и успешно выполнять требуемые тесты в~рамках ограничений бюджета и~планирования
  	\end{itemize}
Косвенная цель:
 	 \begin{itemize}
\item Скомпилировать запись ошибок ПО для~использования при~предотвращении ошибок (путем корректирующих и~профилактических действий)
  	\end{itemize}
\end{frame}
 


\begin{frame} \frametitle{Классификация тестирования ПО}
Тестирование чёрного ящика:
 	 \begin{itemize}
\item Тестирование, которое игнорирует внутренний механизм системы или~компонента и~фокусируется исключительно на~выходах, генерируемых в~ответ на~выбранные входы и~условия выполнения
\item Тестирование проводится для оценки соответствия системы или~компонента указанным функциональным требованиям
  	\end{itemize}
Тестирование белого ящика:
 	 \begin{itemize}
\item Тестирование, которое учитывает внутренний механизм системы или~компонента
  	\end{itemize}
\end{frame}
 


\begin{frame} \frametitle{CASE-средства}
	\begin{block}{}
	\alert{CASE-средства} "--- инструменты автоматизации процессов проектирования и~разработки программного обеспечения
	\end{block}
 	 \begin{itemize}
\item Основным компонентом инструментов CASE является хранилище, в~котором хранится вся информация, связанная с~проектом 
\item В функции CASE входят средства анализа, проектирования и программирования программных средств, проектирования интерфейсов, документирования и производства структурированного кода на каком-либо языке программирования
  	\end{itemize}
\end{frame}



\begin{frame} \frametitle{Вклад CASE в SQA}
 	 \begin{itemize}
\item Определение отклонений от проектных требований
\item Определение конструктивных несоответствий
\item Автоматизированная генерация кода на~основе проектных записей репозитория без~ожидаемых ошибок
\item Полное соответствие инструкциям по~проектной документации и~документации по~кодированию
\item Высококачественные исправления и изменения, внесенные в~процессе разработки
\item Автоматическая генерация репозитория устаревших систем с помощью инструментов обратного преобразования CASE 
  	\end{itemize}
\end{frame}

\lecturenotes

 	 \begin{itemize}
\item Определение отклонений от проектных требований
\item Определение конструктивных несоответствий
\item Автоматизированная генерация кода на~основе проектных записей репозитория без~ожидаемых ошибок
\item Полное соответствие инструкциям по~проектной документации и~документации по~кодированию, достигнутое с~помощью автоматизированного кодирования и~документации на~основе репозитория
\item Высококачественные исправления и изменения, внесенные в~процессе разработки благодаря поддержке инструментов репозитория
\item Автоматическая генерация репозитория устаревших систем с помощью инструментов обратного преобразования CASE, которая позволяет эффективно разрабатывать новые поколения программной системы с максимальной гарантией качества программного обеспечения~\cite[с.~302]{SQA-Galin}
  	\end{itemize}



\begin{frame} \frametitle{Затраты на SQA}
 	 \begin{itemize}
\item Затраты на предотвращение ошибок
\item Затраты на оценку 
\item Затраты на управленческую подготовку и~контроль
\item Затраты на внутреннюю неисправность
\item Затраты на внешний отказ
\item Затраты на управленческую ошибку
  	\end{itemize}
\end{frame}

\lecturenotes

 	 \begin{itemize}
\item Затраты на предотвращение ошибок, т.е. затраты на обучение и подготовку команды технического обслуживания, расходы на профилактические и корректирующие действия
\item Затраты на оценку "--- затраты на обнаружение ошибок, т.е. затраты на обзор услуг основного обслуживания, выполняемых командами SQA, внешними командами и опросами удовлетворенности клиентов
\item Затраты на управленческую подготовку и контроль "--- затраты на управленческую деятельность, проводимую для предотвращения ошибок, т.е. затраты на подготовку планов технического обслуживания, подбор команды обслуживания и последующие действия по техническому обслуживанию
\item Затраты на внутреннюю неисправность "--- затраты на исправления сбоев программного обеспечения, инициированные командой технического обслуживания (до получения жалоб клиентов)
\item Затраты на внешний отказ - Затраты на исправление ошибок программного обеспечения, инициированные жалобами клиентов
\item Затраты на управленческую ошибку "--- расходы на сбои программного обеспечения, вызванные управленческими действиями или бездействием, т.е. расходы на ущерб, вызванные нехваткой обслуживающего персонала и / или недостаточной организацией обслуживания~\cite[с.~272]{SQA-Galin}
  	\end{itemize}


\begin{frame} \frametitle{Корректирующие и~профилактические действия}
Корректирующие действия:
 	 \begin{itemize}
\item\small Применение процесса обратной связи (сбор информации о~несоответствиях качества)
\item\small Выявление и анализ источников нарушений
\item\small Разработка и применение улучшенных методов и~процедур
\item\small Контроль за внедрением методов и~оценка их результатов
  	\end{itemize}
Профилактические действия:
 	 \begin{itemize}
\item\small Применение процесса обратной связи (сбор информации о~потенциальных проблемах качества)
\item\small Выявление и анализ отклонений от~стандартов качества
\item\small Разработка и применение улучшенных методов и~процедур
\item\small Контроль за внедрением методов и~оценка их результатов
  	\end{itemize}
\end{frame}

\lecturenotes

Корректирующие и профилактические действия в процессе разработки и обслуживания ПО~\cite[с.~351]{SQA-Galin}




\section{Обучение и переквалификация персонала}
\begin{frame} \frametitle{Цели обучения и переквалификации персонала}
 	 \begin{itemize}
\item Выполнять задачи по разработке и~обслуживанию ПО на~должном уровне эффективности и~результативности
\item Обеспечить соответствие стандартам организации программных продуктов
\item Обновлять знания и навыки персонала в~ответ на~изменения в организации
\item Передавать знания о процедурах SQA
\item Обеспечить переквалификацию кандидатов на~ключевые должности по~разработке и~обслуживанию ПО
  	\end{itemize}
\end{frame}

\lecturenotes

 	 \begin{itemize}
\item Для развития знаний и навыков новых сотрудников необходимо выполнять задачи по разработке и обслуживанию ПО на должном уровне эффективности и результативности. Такая подготовка способствует интеграции новых членов команды
\item Обеспечить соответствие стандартам организации программных продуктов (документов и кода) путем сопоставления стилевых и структурных процедур с рабочими инструкциями
\item Обновлять знания и навыки персонала в ответ на изменения в организации, и обеспечить действенное и эффективное выполнение задач, а также соответствие стилевых и структурных процедур рабочим инструкциям
\item Передавать знания о процедурах SQA
\item Обеспечить переквалификацию кандидатов на ключевые должности по разработке и обслуживанию ПО~\cite[с.~337]{SQA-Galin}
  	\end{itemize}



\begin{frame} \frametitle{Действия для подготовки программы обучения персонала}
 	 \begin{enumerate}
\item Определить требования к знаниям для~каждой должности
\item Определить потребность в обучении и~профессиональном росте
		\begin{itemize}
	\item Новых сотрудников (обучение)
	\item Сотрудников, назначенных на~новую должность (переподготовка)
	\item Других сотрудников (профессиональный рост)
  		\end{itemize}
\item Спланировать программы обучения и~профессионального роста
  	\end{enumerate}
\end{frame}

\lecturenotes

 	 \begin{enumerate}
\item  Определить требования к знаниям для каждой должности. К ним относятся знания, полученные при приобретении общего профессионального образования с добавлением знаний и навыков, полученных внутри организации, необходимых для работы
\item  Определить потребности в обучении и профессиональном росте. Эти потребности оцениваются путем сопоставления знаний персонала с текущим положением. Они должны быть указаны для трех групп:
 		 \begin{itemize}
	\item Новые сотрудники (обучение)
	\item Сотрудники, назначенные на новую должность (переподготовка)
	\item Другие сотрудники (профессиональный рост)
  		\end{itemize} 
Потребности в обучении и профессиональном росте также должны определяться требованиями к эффективности, основанные на обратной связи, передаваемой различными подразделениями организации
\item  Спланировать программы обучения и профессионального роста. Эти программы будут отвечать следующим вопросам:
	 	 \begin{itemize}
	\item Использование собственных учебных групп и объектов или аутсорсинг
	\item Время проведения учебных мероприятий (по возможности)
	\item Использование программ электронного обучения~\cite[с.~339-341]{SQA-Galin}
  		\end{itemize}
 \end{enumerate}



\begin{frame} \frametitle{Источники данных о результатах обучения}
 	 \begin{itemize}
\item Регулярные показатели производительности
\item Анкеты, заполненные стажерами, их начальством и~другими лицами
\item Анализ выдающихся достижений, а~также сбоев
\item Специализированный обзор программных продуктов, подготовленных квалифицированными и~обученными сотрудниками
  	\end{itemize}
\end{frame}

\lecturenotes

Анализ данных, накопленных после обучения, предоставляет информацию, необходимую для пересмотра программы обучения и сертификации, путем внесения изменений, добавления и удаления идентифицированных видов деятельности и материалов. Источники данных включают:
 	 \begin{itemize}
\item Регулярные показатели производительности, такие как ошибки и статистика производительности, предварительно подготовленные отдельными подразделениями
\item Анкеты, заполненные стажерами, их начальством и другими лицами
\item Анализ выдающихся достижений, а также сбоев
\item Специализированный обзор программных продуктов (документов и кода), подготовленных квалифицированными и обученными сотрудниками~\cite[с.~344]{SQA-Galin}
  	\end{itemize}



\section{Управление SQA}
\begin{frame} \frametitle{Участики организационной структуры SQA}
 	1. Менеджеры
	\begin{itemize}
\item Руководители высшего руководства, особенно исполнительный директор, непосредственно отвечающий за SQA
\item Менеджеры отдела разработки ПО и обслуживания
\item Руководители проектов и руководители групп проектов развития и технического обслуживания
\item Руководители групп тестирования ПО
  	\end{itemize}
\end{frame}

\lecturenotes

В число участников системы SQA входят сотрудники, чьи задачи по обеспечению качества программного обеспечения включают в себя все или часть функций своей должности, а также другие сотрудники, которые вносят свой вклад в систему SQA за пределами своей регулярной должности.

\begin{frame} \frametitle{Участики организационной структуры SQA}
	2. Тестировщики
	\begin{itemize}
\item Участники отдела тестирования ПО
  	\end{itemize}
	3. Специалисты SQA и заинтересованные специалисты
	\begin{itemize}
\item Попечители SQA
\item Члены комитета SQA
\item Участники форума SQA
\item Члены группы подразделений SQA
  	\end{itemize}
\end{frame}


\begin{frame} \frametitle{Основные обязанности руководства по~SQA (1)}
 	 \begin{itemize}
\item Обеспечить качество программных продуктов компании и~обслуживание ПО
\item Сообщать сотрудникам всех уровней о~важности качества продукции и~услуг 
\item Обеспечить удовлетворительное функционирование и~полное соответствие ПО с~требованиями клиента
\item Обеспечить, чтобы цели качества для~SQA были установлены и~выполнены
  	\end{itemize}
\end{frame}

\begin{frame} \frametitle{Основные обязанности руководства по~SQA (2)}
 	 \begin{itemize}
\item Инициировать планирование и контролировать внедрение изменений, необходимых для~адаптации системы SQA к~внутренним и внешним изменениям
\item Вмешательство непосредственно в~поддержку разрешения спорных ситуаций и~сведение ущерба к~минимуму 
\item Обеспечить доступность ресурсов, необходимых для~систем SQA
  	\end{itemize}
\end{frame}

\lecturenotes

 	 \begin{itemize}
\item Обеспечить качество программных продуктов компании и~обслуживание ПО
\item Сообщать сотрудникам всех уровней о~важности качества продукции и~услуг (в дополнение к удовлетворенности клиентов) 
\item Обеспечить удовлетворительное функционирование и~полное соответствие ПО с~требованиями клиента
\item Обеспечить, чтобы цели качества для~SQA были установлены и~выполнены
\item Инициировать планирование и контролировать внедрение изменений, необходимых для адаптации системы SQA к основным внутренним и внешним изменениям, связанным с~клиентами, конкуренцией и~технологиями организации
\item Вмешательство непосредственно в поддержку разрешения спорных ситуаций и~сведение ущерба к~минимуму 
\item Обеспечить доступность ресурсов, необходимых для систем SQA~\cite[с.~544]{SQA-Galin}
  	\end{itemize}



\begin{frame} \frametitle{Основные обязанности ответственного должностного лица}
 	 \begin{itemize}
\item Нести ответственность за подготовку годовой программы и бюджета SQA для~окончательного утверждения старшим руководством
\item Нести ответственность за подготовку планов развития SQA для~реагирования на изменения во внутренней и~внешней среде организации
\item Иметь общий контроль за реализацией ежегодной программы SQA по планированию деятельности и~проектов развития SQA
\item Представлять и защищать вопросы SQA исполнительному руководству организации
  	\end{itemize}
\end{frame}

\lecturenotes

Источник~\cite[с.~552]{SQA-Galin}



\begin{frame} \frametitle{Политика качества ПО в~компании~(1)}
	\begin{itemize}
\item Присвоение максимального приоритета удовлетворенности клиентов путем оперативного выполнения требований и~ожиданий, запросов и~жалоб
\item Вовлечение сотрудников в определение целей качества и~приверженность их достижению
\item Правильное выполнение задач по разработке и~обслуживанию 
\item Минимизация необходимости доработки и~исправления
\item Обеспечение высокого и соответствующего профессионального и~управленческого уровня своих сотрудников
  	\end{itemize}
\end{frame}

\lecturenotes

Политика качества компании включает в себя:
	\begin{itemize}
\item Присвоение максимального приоритета удовлетворенности клиентов путем оперативного выполнения требований и ожиданий, запросов и жалоб
\item Вовлечение сотрудников в определение целей качества и приверженность их достижению
\item Правильное выполнение задач по разработке и обслуживанию 
\item Минимизация необходимости доработки и исправления
\item Обеспечение высокого и соответствующего профессионального и управленческого уровня своих сотрудников, ценность поддерживается путем предоставления стимулов и поощрения для своих сотрудников для достижения профессионального совершенства
  	\end{itemize}


\begin{frame} \frametitle{Политика качества ПО в~компании~(2)}
	\begin{itemize}
\item Выполнение мероприятий по обеспечению качества на~протяжении всего жизненного цикла ПО для~обеспечения достижения требуемых целей качества
\item Применение стандартов качества для~субподрядчиков и поставщиков
\item Стремление постоянно улучшать производительность разработки и~обслуживания, а также эффективность и~результативность SQA
\item  Распределение всех организационных, физических и~профессиональных ресурсов, необходимых для реализации целей обеспечения качества ПО
  	\end{itemize}
\end{frame}

\lecturenotes

	\begin{itemize}
\item Выполнение мероприятий по обеспечению качества на протяжении всего жизненного цикла ПО для обеспечения достижения требуемых целей качества
\item Применение стандартов качества для субподрядчиков и поставщиков. Только те, которые квалифицируются, будут включены в проекты развития и техническое обслуживание компании
\item Стремление постоянно улучшать производительность разработки и обслуживания, а также эффективность и результативность SQA
\item  Распределение всех организационных, физических и профессиональных ресурсов, необходимых для реализации целей обеспечения качества ПО~\cite[с.~545--546]{SQA-Galin}
  	\end{itemize}









\begin{thebibliography}{99}
\bibitem{SQA-Galin} Galin, Daniel, Software quality assurance: from theory to implementation / Daniel Galin, 2004."--- 590с.: ил.
\end{thebibliography}

\end{document}

%%% Local Variables: 
%%% mode: TeX-pdf
%%% TeX-master: t
%%% End: 
