\documentclass{../industrial-development}
\graphicspath{{template/}}

\title{Процесс превращения разработчика в менеджера}
\author{Румянцева Мария Сергеевна, \\Феофанова Алла Александровна, \\ПИ-21 МО}
\date{}

\begin{document}

\begin{frame}
  \titlepage
\end{frame}

\section{Разработчик и менеджер}

\begin{frame} \frametitle{IT-менеджер}
	IT-менеджеры — это сотрудники, отвечающие за качество услуг, предоставляемых пользователям (сотрудникам компании, а иногда — её клиентам и партнёрам) на базе информационных технологий.
\end{frame}
\lecturenotes
IT-менеджер – это специалист, чьей главной задачей является управление проектом в целом: проектирование и расстановка приоритетов, планирование выполнения задач, контроль, коммуникации, а также оперативное решение проблем.
IT-менеджер компании – сотрудник, управляющий информационными процессами. Он разбирается не только в технических аспектах IT-среды, но и в вопросах ее взаимодействия с другими сферами: финансовой, кадровой, рыночной.  Задача менеджера внутри фирмы: знать цели развития бизнеса компании, уметь представить бизнес-процессы для их автоматизации, осуществить правильный выбор информационной системы и рассчитать эффект от ее применения.  Его внешние цели – обеспечить клиентам получение качественных ИТ-услуг или организовать продажу информационных продуктов.
~\cite{How_to_be_a_good_IT-manager}

\begin{frame} \frametitle{Разработчик}
Разработчик — программист, который занимается созданием различных продуктов в ИТ: компьютерных игр, мобильных приложений, веб-сайтов и др. 
\end{frame}
\lecturenotes
Специфика деятельности разработчика (другое название этой профессии — Developer) всецело зависит от выбранного направления. К примеру, разработкой программного обеспечения прикладного характера (игры‚ бухгалтерские программы‚ редакторы‚ мессенджеры, ПО для систем видео- и аудионаблюдения) занимаются прикладные программисты; созданием операционных систем, работой с сетями, написанием интерфейсов к базам данных озадачены системные программисты; воплощением в жизнь проектов веб-дизайнеров, т. е. созданием сайтов, заняты веб-программисты.
~\cite{How_to_be_a_good_IT-manager}

\begin{frame} \frametitle{Важные качества разработчика в IT}
	 \begin{itemize}
	 	\item Сильные технические навыки
		\item Аналитический склад ума
	 	\item Готовность к обучению
	 	\item Умение работать в команде
		\item Стрессоустойчивость
		\item Вовлеченность в рабочий процесс
		\item Способность к концентрации
	 \end{itemize}
\end{frame}
\lecturenotes
Сфера информационных технологий и телекоммуникаций отличается стремительным развитием, и для сохранения набранного темпа становятся всё необходимей специалисты, которые помогли бы ей в этом. В то же время, специалистам следует соответствовать определённым требованиям, без них они не смогут успешно справляться со своими профессиональными задачами. Так какие же личные качества помогают работникам в сфере IT и телекоммуникаций стать настоящими профессионалами?
Такие специалисты в первую очередь должны быть склонны к математике, информатике и работе с техникой, в том числе с компьютерами. Им необходимо иметь аналитический склад ума, хорошую память и способность работать с большим количеством информации. Также незаменимыми качествами для всех сотрудников в этой области, независимо от должности, являются ответственность, организованность, стрессоустойчивость, умение самостоятельно обучаться по специализированной литературе.
Стоит отметить, что облик IT-специалиста, работавшего, к примеру, десять лет назад, существенно отличается от современного. Теперь это не молчаливый, сосредоточенный человек, не отрывающийся от компьютера весь рабочий день, а коммуникабельный сотрудник, готовый к работе в команде и прямому диалогу с клиентами. Создавая продукт, работники IT-сферы и телекоммуникации ориентируются на его будущих потребителей, поэтому хорошо знают интересы и потребности своих потенциальных клиентов.
Важно не просто иметь все эти качества, но и грамотно себя преподносить, например, во время трудоустройства. Главная визитная карточка здесь – резюме, в нём стоит обязательно указать свои преимущества и способности. 
У обладателей таких качеств действительно много шансов найти работу по своей специальности, тем более что спрос на работников в сфере информационных технологий и телекоммуникаций заметно превышает предложение. Конечно, для того чтобы быть компетентными и востребованными им понадобится не только психологическая склонность к профессии. Им не обойтись без хорошего образования, профессиональных знаний и умений, постоянного совершенствования своих навыков. А знание английского языка, к тому же, поможет найти работу в уже известной или очень перспективной зарубежной компании.
~\cite{How_to_be_a_good_IT-manager}

\subsection{Ранги в профессии программист}
\begin{frame} \frametitle{Ранги в профессии программист}

	 	\begin{table}[H]
\caption{\label{tab:canonsummary} Ступени профессионального роста разработчика}
\begin{center}
\begin{tabular}{|p{0.2\linewidth}|p{0.8\linewidth}|}
\hline
\textbf{Должность} & \textbf{Специфика} \\
\hline
Junior Developer &  Младший программист – это молодой разработчик с малым количеством опыта. \\
\hline
Developer или Middle Dev  & Разработчик -  это человек, который поддерживает Junior разработчиков, занимается как архитектурой проектов, так и модульной реализацией.   \\
\hline
Senior developer & Ведущий программист – человек, который выполняет такие работы, как: детальное проектирование и создание спецификаций проектов, полностью контролирует проектирование проектов. \\
\hline
\end{tabular}
\end{center}
\end{table} 
\end{frame}
\lecturenotes
Младший программист – Junior Developer, это молодой разработчик с малым количеством опыта или вовсе без такового, только начавший работу в избранной технологичной области. Обучается, как правило, по видеокурсам и видеоурокам, причём постоянно ведь опыта разработки у него так мало, что он очень много времени тратит на советы с более опытным разработчиком и понимание того как команда работает  над проектом. Накопив достаточно опыта и реализовав несколько проектов, junior переходит на middle уровень – становясь полноценным разработчиком. Но всё же, Junior разработчики находятся в постоянном, непрерывном и самостоятельном образовательном процессе. Обучение для junior разработчика -  это все. Испытывая постоянную необходимость в выполнении задач, поставленных по работе над проектом, идеальная схема обучения для Junior разработчика - это видео обучение программированию в рамках избранной технологии. Опираясь на просмотренные видео уроки по программированию, онлайн консультации и советы более опытных коллег, junior разработчик растёт достаточно быстро для того чтобы уже через 1-1,5 года занять позицию полноценного разработчика. Требования, обычно выставляемые к Junior разработчику, приблизительно таковы, как и у Developer и Senior.

Developer или Middle Dev
Программист или Developer или Middle dev (англ. разработчик) - это человек, ответственный за качественное и своевременное исполнение разработки информационно-программных систем, основанных на применении современных программных технологий. Программист выполняет задачи по написанию и базовому тестированию порученных ему компонентов системы, работает developer по внешним спецификациям. Поддерживает Junior разработчиков, занимается как архитектурой проектов, так и модульной реализацией, производит реализацию работоспособности прототипов, постоянно занимается самообразованием, понимает алгоритмы, Software Engineering Process, обладает знаниями в следующих областях: языки разметки, понимание технологии web-серверов и серверов приложений, знанием клиентских и серверных технологий , работы браузера, СУБД, операционных систем, офисных пакетов, сред разработки, профильных языков программирования, технического английского. Как правило, имеет высшее образование, хотя случается и что не имеет, это не критичный показатель, критичный -  наличие знаний и опыта.

 Ведущий программист – человек ответственный за качество и своевременность работ по разработке информационно-программных систем, основанных на применении новейших программных технологий. Обладает глубокими, структурированными знаниями и работает внутри проектной команды, совершенно не имея необходимости контактировать с представителями менеджмента заказчика. Выполняет такие работы, как: детальное проектирование и создание спецификаций проектов, полностью контролирует и зачастую и самостоятельно выполняет проектирование мелких проектов и внутренних под-проектов (модулей), программирование и базовое тестирование компонентов. Как правило, имеет законченное высшее образование, реже незаконченное, стаж от 3х лет в качестве developer, умеет комментировать программы, не прибегая к использованию словаря, разрабатывать документацию, свободно общаться на английском языке, владеет методами и инструментами анализа и проектирования, Software Engineering Process, языками разметки, глубоким пониманием клиент-сервер технологии, работ браузера, web серверов, серверов приложений, БД, ОС, офисными пакетами, может контролировать других разработчиков и ставить им задачи.
~\cite{How_to_be_a_good_IT-manager}

\subsection{Основания для перевода разработчика в менеджера}
\begin{frame} \frametitle{Основания компании для перевода разработчика в менеджеры}
\begin{itemize}
		\item Отсутствие менеджера
		\item Дефицит времени на введение в работу человека со стороны
		\item Желание "выращивать" менеджера внутри компании
		\item Нежелание увеличивать штат сотрудеников
	\end{itemize}
\end{frame}
\lecturenotes
???
~\cite{How_to_be_a_good_IT-manager}

\begin{frame} \frametitle{Причины, по которым разработчику предлагают должность менеджера}
\begin{itemize}
		\item Лидерские качества
		\item Умение находить общий язык
		\item Знание процессов разработки и особенностей продукта изнутри
		\item Готовность взять ответственность не только за себя, но и за других
		\item Желание развиваться
		\item Доверие (авторитет) у заказчика
		\item Достижение определенных профессиональных высот

	\end{itemize}
\end{frame}
\lecturenotes
???
~\cite{How_to_be_a_good_IT-manager}

\subsection{Подготовка менеджера проекта}
\begin{frame} \frametitle{Подготовка менеджера проекта}

	 	\begin{table}[H]
\caption{\label{tab:canonsummary} Обучение специалистов}
\begin{center}
\begin{tabular}{|p{0.2\linewidth}|p{0.8\linewidth}|}
\hline
\textbf{Вид} & \textbf{Специфика} \\
\hline
Официаль-ные &  обучение специалистов в ВУЗ-ах и на специальных курсах, завершение которых удостоверяется соответствующим документом. \\
\hline
Полуофи-циальные  & прохождение краткосрочных курсов, посещение популярных лекций и практических занятий. \\
\hline
Неофициаль-ные &  участие в конференциях, семинарах и ознакомление с соответствующей литературой \\
\hline
Обучение в процессе работы & обучение на рабочем месте, а также самообразование. \\
\hline
\end{tabular}
\end{center}
\end{table} 
\end{frame}
\lecturenotes
официальные – обучение специалистов в ВУЗ-ах и на специальных курсах, завершение которых удостоверяется соответствующим документом. К слушателям этой формы подготовки предъявляются обязательные требования, как при начале, так и при завершении обучения, а в ряде случаев, и по ходу самого обучения;
полуофициальные – прохождение насыщенной программы обучения на краткосрочных курсах (продолжительностью от нескольких дней до нескольких месяцев), посещение популярных лекций и практических занятий. К слушателям этой формы подготовки не предъявляются какие-либо обязательные требования, и они не получают специальных удостоверений об их окончании;
неофициальные – участие в конференциях, симпозиумах, региональных семинарах, собраниях профессиональных обществ, а также ознакомление с соответствующей литературой;
обучение в процессе работы – это обучение на рабочем месте при выполнении конкретного проекта, а также самообразование.
~\cite{How_to_be_a_good_IT-manager}
\subsection{Отличия стиля мышления и характера работы}
\begin{frame} \frametitle{Отличия стиля мышления разработчика и менеджера}

\end{frame}
\lecturenotes

~\cite{How_to_be_a_good_IT-manager}

\begin{frame} \frametitle{Отличия характера работы разработчика и менеджера}

\end{frame}
\lecturenotes

~\cite{How_to_be_a_good_IT-manager}

\begin{frame} \frametitle{Становление менеджера из разработчика}
	\begin{block}{Предпочтение разработчику при выборе менеджере}
	 \begin{itemize}
	 	\item 
	 	\item 
	 	\item 
	 \end{itemize}
 	\end{block}
\end{frame}
\lecturenotes

~\cite{How_to_be_a_good_IT-manager}


\begin{frame} \frametitle{Становление менеджера из разработчика}
	\begin{block}{Недостатки менеджера вышедшего из разработчика}
	 \begin{itemize}
	 	\item 
	 	\item 
	 	\item 
	 \end{itemize}
 	\end{block}
\end{frame}
\lecturenotes

~\cite{How_to_be_a_good_IT-manager}

\begin{frame} \frametitle{Ошибки начинающего менеджера}
\begin{itemize}		
		\item Стремление выполнить всю работу самостоятельно
	 	\item Узкое мышление
	 	\item Уход от ответственности за работу всей команды
		\item Чрезмерный контроль над работой подчиненных
	 	\item Неправильное распределение обязанностей
	 	\item Постановка нереальных сроков для выполнения задач
 \end{itemize}
\end{frame}
\lecturenotes

~\cite{How_to_be_a_good_IT-manager}

\subsection{Этапы развития карьеры менеджера в компании}

\begin{frame} \frametitle{Этапы развития карьеры менеджера в компании}

\end{frame}
\lecturenotes

~\cite{How_to_be_a_good_IT-manager}

\section{Организация рабочего процесса менеджера}


\begin{frame} \frametitle{}

\end{frame}
\lecturenotes

~\cite{How_to_be_a_good_IT-manager}

\begin{frame} \frametitle{}

\end{frame}
\lecturenotes

~\cite{How_to_be_a_good_IT-manager}





\begin{thebibliography}{99}
\bibitem{How_to_be_a_good_IT-manager} \href{http://www.pvsm.ru/upravlenie-proektami/36476}
\bibitem{Managers_in_IT} \href{https://habrahabr.ru/post/219741/}
\bibitem{Managers_thinking_style} \href{http://www.hr-asteri.ru/employer/poleznaya_informaciya/stil_myshleniya_i_povedeniya_professionalnogo_rukovoditelya/}
\bibitem{Programmer_vs_manager} \href{https://dou.ua/lenta/articles/programmer-vs-manager/}
\bibitem{Best_qualities_for_IT-manager} \href{http://hr-portal.ru/article/kakie-kachestva-nuzhny-menedzheru-it-proektov}
\bibitem{From_engineer_to_manager_keeping_your_technical_skills} \href{https://hackernoon.com/from-engineer-to-manager-keeping-your-technical-skills-40579cc8ea00}
\bibitem{From_programmer_to_manager} \href{https://m.dotdev.co/what-i-learned-transitioning-from-being-a-programmer-to-an-it-manager-8e58e7b406}
\end{thebibliography}

\end{document}

%%% Local Variables: 
%%% mode: TeX-pdf
%%% TeX-master: t
%%% End: 
